\documentclass[12pt, a4paper]{article}

\usepackage[hidelinks]{hyperref}
\usepackage[tmargin=1in]{geometry}
\usepackage{parskip}
\usepackage{amssymb}
\usepackage{amsmath}
\usepackage[shortlabels]{enumitem}
\usepackage[dutch]{babel}
\selectlanguage{dutch}

\begin{document}

\title{Bewijzen - Inleveropgave 1}
\author{B.H.J. van Boxtel}
\date{21 september 2022 - Week 38} 

\maketitle
\pagenumbering{gobble} %is om de paginanummering pas te laten starten na de titelpagina%

\begin{enumerate}[(a).] 
    \item $I = \{[\, 3 - \frac{1}{n}, 6 \,]\}_{n \in \mathbb{N}}$
    \bigskip
    \item $n \neq m \implies \frac{1}{n} \neq \frac{1}{m} \implies [3 -\frac{1}{n}, 6] \neq [3 - \frac{1}{m}] \implies A_n \neq A_m$.
    \bigskip
    \item Stel $n,m \in \mathbb{N}$ met $n < m$ en $n \neq m$.
    
    Dan $n < m \implies \frac{1}{n} > \frac{1}{m} \implies 3 - \frac{1}{n} < 3 - \frac{1}{m}$.

    Dus $A_m \subseteq A_n$.

    Omdat $1 \leq x$ voor alle $x \in \mathbb{N}$, $ A_m \subseteq A_1$ voor elke $m \in \mathbb{N}$.

    Dus $\bigcup_{n \in \mathbb{N}} A_n \subseteq A_1 \iff \bigcup_{n \in \mathbb{N}} A_n \subseteq [2,6]$.
    \bigskip

    $n = 1 \implies A_n = [2,6]$, dus $[2,6] \subseteq \bigcup_{n \in \mathbb{N}} A_n$.

    Dus $\bigcup_{n \in \mathbb{N}} A_n = [2,6]$.
    \bigskip

    \item Er bestaat geen getal $n \in \mathbb{N}$ waarvoor $3 - \frac{1}{n} > 3$, dus $[3,6] \subseteq A_n$ voor elke $n \in \mathbb{N}$. Dus $[3,6] \subseteq \bigcap_{n \in \mathbb{N}} A_n$.
    \bigskip

    \item Omdat $\lvert \mathbb{N} \rvert = \infty$, geldt:
    \begin{equation} \label{eq:1}
        \bigcap_{n \in \mathbb{N}} A_n = \lim_{a\to\infty}\bigcap_{k = 1}^{a} A_k
    \end{equation}

    Neem nu $n,m \in \mathbb{N}$ met $n < m$ en $n \neq m$.

    $n < m \implies 3 - \frac{1}{n} < 3 - \frac{1}{m} \implies [3 - \frac{1}{m}, 6] \subseteq [3 - \frac{1}{n}, 6]$ en $[3 - \frac{1}{n}, 6] \nsubseteq [3 - \frac{1}{m}, 6]$
    
    $\implies [3 - \frac{1}{m}, 6] \bigcap \, [3 - \frac{1}{n}, 6] = [3 - \frac{1}{m}, 6].$ 
    
    Dus $n < m \implies A_m \bigcap A_n = A_m$ $\implies \bigcap_{k = 1}^{a} A_k = A_a$.
    \bigskip

    Samen met (\ref{eq:1}) geeft dit:
    \begin{equation}
        \bigcap_{n \in \mathbb{N}} A_n = \lim_{a\to\infty}A_a = \lim_{a\to\infty} [3 - \frac{1}{a}, 6] = [3 , 6]
    \end{equation}

\end{enumerate}


\end{document}