\documentclass[12pt, dutch, a4paper]{article}

\usepackage[hidelinks]{hyperref}
\usepackage[tmargin=0.8in,bmargin=1in]{geometry}
\usepackage{parskip}
\usepackage{amssymb}
\usepackage{amsmath}
\usepackage[shortlabels]{enumitem}
\usepackage[dutch]{babel}
\selectlanguage{dutch}
\usepackage{amsthm}
\usepackage{cleveref}

\theoremstyle{definition}

\newtheorem{theorem}{Theorem}
\newtheorem{lemma}{Lemma}[theorem]
\newtheorem{sublemma}{Lemma}[lemma]
\newtheorem{case}{Geval}

\newenvironment{shortthm}
  {\refstepcounter{theorem}\textbf{Theorem~\thetheorem.}}% \begin{shortthm}
{\enskip}

\begin{document}

\title{Bewijzen - Inleveropgave 1 - Poging 2}
\author{B.H.J. van Boxtel}
\date{5 Oktober 2022 - Week 40} 

\maketitle
\pagenumbering{arabic} 

\begin{enumerate}[(a.)]
    \item
    Een voorbeeld van een ge\"indiceerde collectie $I$ 
    die voldoet aan de volgende vereisten:
    \begin{itemize}
        \item Alle $A_n$ zijn verschillend.
        \item $\bigcup_{n \in \mathbb{N}}A_n = [2,6]$.
        \item $\bigcap_{n \in \mathbb{N}}A_n = [3,6]$.
    \end{itemize}

    Is $I = \{A_n\}_{n \in \mathbb{N}}$, 
    waar $A_n = [3 - \frac{1}{n}, 6]$.
    \bigskip

    \item 
    \begin{shortthm} \label{theoremB}
        De verzamelingen $A_n$ zijn paarsgewijs verschillend. In andere woorden 
        $n \neq m \implies A_n \neq A_m$. 
    \end{shortthm}
    \begin{proof}[Bewijs van \textbf{Theorem \ref{theoremB}.}] $ $ \newline
        Neem $n,m \in \mathbb{N}$ zodat $n \neq m$. \newline
        Dan is $\tfrac{1}{n}$ niet gelijk aan $\tfrac{1}{m}$, dus 
        $3 - \tfrac{1}{n}$ is niet gelijk aan $3 - \tfrac{1}{m}$.

        Omdat de linkergrenzen van de intervallen niet gelijk zijn, 
        zijn de intervallen niet gelijk.

        $[3 - \tfrac{1}{n}, 6] = A_{n} \neq [3 - \tfrac{1}{m}, 6] = A_{m}$. \newline
        Dus $n \neq m \implies A_n \neq A_m$. \newline
    \end{proof}    
    \bigskip

    \item 
    \begin{shortthm} \label{theoremC}
        De vereniging van de collectie $I$ is gelijk aan het interval $[2,6]$. 
        In andere woorden: $\bigcup_{n \in \mathbb{N}}A_n = [2,6]$.
    \end{shortthm}

    Hiervoor is een kort subargument nodig.
    \begin{lemma} \label{lemmaC}
        Voor een gegeven $n$ en $m$ zodat $n < m$ geldt dat $A_m \subseteq A_n$. 
    \end{lemma}

    \begin{proof}[Bewijs van \textbf{\cref{lemmaC}}] $ $ \newline
        Neem $n,m \in \mathbb{N}$ zodat $n < m$. \newline
        Dan geldt $\tfrac{1}{n} > \tfrac{1}{m}$, 
        en ook $3 - \tfrac{1}{n} < 3 - \frac{1}{m}$.

        Vanuit de definitie van $A_n$ zien we dat $3-\frac{1}{n}$ de linkergrens van 
        $A_n$ is en $3 - \frac{1}{m}$ de linkergrens van $A_m$.
        Omdat de linkergrens van $A_n$ nu kleiner is dan de linkergrens van $A_m$
        en de rechtergrens van beide hetzelfde is, 
        is het interval $A_m$ een deelverzameling van het interval $A_n$. 

        Dus $n < m \implies A_m \subseteq A_n$. \newline 
    \end{proof} 

    \newpage
    Nu kan \textbf{Theorem \ref{theoremC}.} bewezen worden.
    \begin{proof}[Bewijs van \textbf{Theorem \ref{theoremC}}] $ $ \newline
        Omdat $1 \leq n$ voor elke $n \in \mathbb{N}$, 
        en als gevolg van \textbf{\cref{lemmaC}}  
        geldt $A_n \subseteq A_1$ voor elke $n$.
        
        $A_1 = [2,6]$,
        dus $\bigcup_{n \in \mathbb{N}}A_n \subseteq A_1
        \iff \bigcup_{n \in \mathbb{N}}A_n \subseteq [2,6]$. \newline
    \end{proof}
    \bigskip

    \item 
    \begin{shortthm} \label{theoremD}
        Het interval $[3,6]$ is een deelverzameling van $\bigcap_{n \in \mathbb{N}}A_n$.
    \end{shortthm}
    \begin{proof}[Bewijs van \textbf{Theorem \ref{theoremD}.}] $ $ \newline
        Voor elke $n \in \mathbb{N}$ geldt $\tfrac{1}{n} > 0$. \newline
        Dus $-\tfrac{1}{n} < 0$, dus $3 - \tfrac{1}{n} < 3$.

        Dus er bestaat geen $n \in \mathbb{N}$ zodat $3 - \tfrac{1}{n} > 3$. \newline
        Dus $[3,6] \subseteq A_n$ voor elke $n$.

        Dus $[3,6] \subseteq \bigcap_{n \in \mathbb{N}}A_n$. \newline
    \end{proof}
    \bigskip

    \item 
    \begin{shortthm} \label{TheoremE}
        De doorsnede van de collectie $I$ is een deelverzameling van het interval $[3,6]$.
        In andere woorden: $\bigcap_{n \in \mathbb{N}}A_n \subseteq [3,6]$.
    \end{shortthm}

    Ook hier begin ik met het bewijzen van een subargument.

    \begin{lemma} \label{LemmaE}
        Voor een gegeven $n$ en $m \in \mathbb{N}$ zodat $n < m$, 
        geldt dat de doorsnede van $A_n$ en $A_m$ gelijk is aan $A_m$.  
        In andere woorden: \newline
        $n < m \implies A_n \bigcap A_m = A_m$.
    \end{lemma}

    \begin{proof}[Bewijs van \textbf{\cref{LemmaE}}]
        Neem $n,m \in \mathbb{N}$ met $n < m$. \newline
        Dan geldt $3 - \frac{1}{n} < 3 - \frac{1}{m}$.

        Dus $[3 - \frac{1}{m}, 6] \subseteq [3 - \frac{1}{n}, 6]$ en 
        $[3 - \frac{1}{n}, 6] \nsubseteq [3 - \frac{1}{m}, 6]$.

        Dus $A_m$ is een subset van $A_n$, en $A_n$ is geen subset van $A_m$. \newline
        Dus $A_n \bigcap A_m = A_m$.

        Dus $n < m \implies A_n \bigcap A_m = A_m$. \newline
    \end{proof}
    \bigskip

    \begin{proof}[Bewijs van \textbf{Theorem \ref{TheoremE}}] $ $ \newline
        Omdat $\lim_{x\to\infty} 3-\tfrac{1}{a} = 3$ 
        is 3 de bovengrens van de linkergrens van het interval. In andere woorden,
        de linkergrens van $A_n$ kan niet groter worden dan 3.

        Omdat ook elke $A_n$ uniek is, 
        betekent dit dat het interval $[3,6]$ het enige interval is 
        dat een deelverzameling is van alle $A_n$.

        Dus $\bigcap_{n \in \mathbb{N}}A_n = [3,6]$. \newline
    \end{proof}
\end{enumerate}



\end{document}