\documentclass[12pt, dutch, a4paper]{article}

\usepackage[hidelinks]{hyperref}
\usepackage[tmargin=1in]{geometry}
\usepackage{parskip}
\usepackage{amssymb}
\usepackage{amsmath}
\usepackage[shortlabels]{enumitem}
\usepackage[dutch]{babel}
\selectlanguage{dutch}
\usepackage{amsthm}
\usepackage{cleveref}

\theoremstyle{definition}

\newtheorem{theorem}{Theorem}
\newtheorem{lemma}{Lemma}[theorem]
\newtheorem{sublemma}{Lemma}[lemma]
\newtheorem{case}{Geval}

\begin{document}

\title{Bewijzen - Inleveropgave 2}
\author{B.H.J. van Boxtel}
\date{28 september 2022 - Week 39} 

\maketitle
\pagenumbering{gobble} %is om de paginanummering pas te laten starten na de titelpagina%

\begin{theorem} \label{maintheorem}
    $n^2 \equiv 1 \pmod{3} \iff 3 \nmid n$ voor elke $n \in \mathbb{Z}$. 
\end{theorem}   

Om deze stelling te bewijzen moeten allebei de implicaties 
(van links naar rechts en van rechts naar links) worden bewezen. Deze zal ik apart
als \textbf{\cref{implicatie1lemma}} en \textbf{\cref{implicatie2lemma}} bewijzen.

\begin{lemma} \label{implicatie1lemma}
    $n^2 \equiv 1 \pmod{3} \implies 3 \nmid n$ voor elke $n \in \mathbb{Z}$. 
\end{lemma}
Voor het bewijs van deze implicatie heb ik nog een supplementeel argument nodig, 
wat ik als eerst bewijs. 

\begin{sublemma} \label{sublemma1}
    $3 \nmid n^2 \implies 3 \nmid n$.
\end{sublemma}

Dit ga ik bewijzen met behulp van contrapositie, in andere woorden ik ga bewijzen dat
$3 \: | \: n \implies 3 \: | \: n^2$.
\begin{proof} $ $\newline
    Stel $3 \: | \: n$ met $n \in \mathbb{Z}$. \newline
    Dan is $n$ te schrijven als $n = 3q$ met $q \in \mathbb{Z}$. \newline
    Als we links en rechts kwadrateren, 
    vinden we $n^2 = 9q^2$. \newline
    Dit kunnen we ook schrijven als $n^2 = 3 \cdot 3q^2$. \newline
    $3q^2 \in \mathbb{Z}$, dus hieruit volgt dat $3 \: | \: n^2$. \newline 
    Dus $3 \: | \: n \implies 3 \: | \: n^2$. \newline
    Dus $3 \nmid n^2 \implies 3 \nmid n$.
\end{proof}

Met \textbf{\cref{sublemma1}} bewezen, 
kan \textbf{\cref{implicatie1lemma}} worden bewezen.
\begin{proof} $ $\newline
    Stel $n^2 \equiv 1 \pmod{3}$. \newline
    Dit betekent dat $n^2$ gelijk is aan $3k+1$ met $k \in \mathbb{Z}$. \newline
    $3 \nmid 3k + 1$, dus $3 \nmid n^2$. \newline
    Volgens \textbf{\cref{sublemma1}} betekent dit dat $3 \nmid n$. \newline
    Dus $n^2 \equiv 1 \pmod{3} \implies 3 \nmid n$ voor elke $n \in \mathbb{Z}$.
\end{proof}

\newpage
\begin{lemma} \label{implicatie2lemma}
    $3 \nmid n \implies n^2 \equiv 1 \pmod{3} $ voor elke $n \in \mathbb{Z}$. 
\end{lemma}

Dit zal ik met behulp van gevallenonderzoek bewijzen. 
Als $n$ niet deelt door 3, geeft dit twee gevallen, een waar $n$ 1 meer is dan 
een veelvoud van 3, en een waar $n$ 2 meer is dan een veelvoud van 3. Voor elk 
van deze gevallen bewijs ik dat hieruit volgt dat $n^2 \equiv 1 \pmod{3}$.

\begin{proof} 
    \begin{case} $n = 3k + 1$ met $k \in \mathbb{Z}$. \newline
        $n = 3k + 1 \implies n^2 = 9k^2 + 6k + 1 = 3(3k^2 + 2k) + 1$ \newline
        $k \in \mathbb{Z}$, dus $3k^2 + 2k \in \mathbb{Z}$. \newline 
        Na delen door 3 blijft er dus 1 als rest over. \newline 
        Dit betekent $n^2 \equiv 1 \pmod{3}$.
    \end{case}
    \begin{case} $n = 3k + 2$ met $k \in \mathbb{Z}$. \newline
        $n = 3k + 2 \implies n^2 = 9k^2 + 12k + 1 = 3(3k^2 + 6k) + 1$ \newline
        $k \in \mathbb{Z}$, dus $3k^2 + 6k \in \mathbb{Z}$. \newline 
        Na delen door 3 blijft er dus 1 als rest over. \newline 
        Dit betekent $n^2 \equiv 1 \pmod{3}$.
    \end{case}
    Dus $n^2 \equiv 1 \pmod{3}$ geldt voor elk geval. \newline
    Dus $3 \nmid n^2 \implies n^2 \equiv 1 \pmod{3}$.
\end{proof}

Nu kan \textbf{Theorem \ref{maintheorem}} bewezen worden.
\begin{proof} $ $ \newline
    Omdat alle twee de implicaties 
    (\textbf{\cref{implicatie1lemma}} en \textbf{\cref{implicatie2lemma}})
    nu zijn bewezen, betekent dit dat
    $n^2 \equiv 1 \pmod{3} \iff 3 \nmid n$ voor elke $n \in \mathbb{Z}$.
\end{proof}



\end{document}