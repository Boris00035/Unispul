\documentclass[12pt, dutch, a4paper]{article}

\usepackage[hidelinks]{hyperref}
\usepackage[tmargin=0.8in,bmargin=1in]{geometry}
\usepackage{parskip}
\usepackage{amssymb}
\usepackage{amsmath}
\usepackage[shortlabels]{enumitem}
\usepackage[dutch]{babel}
\selectlanguage{dutch}
\usepackage{amsthm}
\usepackage{cleveref}

\theoremstyle{definition}

\newtheorem{theorem}{Stelling}
\newtheorem{lemma}{Lemma}[theorem]
\newtheorem{lemmalos}{Lemma}
\newtheorem{sublemma}{Lemma}[lemma]
\newtheorem{case}{Geval}
\newtheorem{claim}{Claim}

\newenvironment{shortclaim}
  {\refstepcounter{claim}\textbf{Claim~\theclaim.}}% \begin{shortthm}
{\enskip}
\newenvironment{shortthm}
  {\refstepcounter{theorem}\textbf{Stelling~\thetheorem.}}% \begin{shortthm}
{\enskip}

\begin{document}

\title{Bewijzen - Inleveropgave 3}
\author{B.H.J. van Boxtel}
\date{5 Oktober 2022 - Week 40} 

\maketitle
\pagenumbering{arabic} 

Gegeven is het volgende lemma:
\begin{lemmalos}\label{lemma1}
    Zij $m,n \in \mathbb{Z}$ en $p$ een priemgetal. Als $p \mid mn$, 
    dan geldt dat $p \mid m$ of $\; p \mid n$.
\end{lemmalos}

\begin{enumerate}[(a.)]
    \item
    \begin{shortthm}\label{TheoremA}
        Voor twee verschillende priemgetallen $p,q$ geldt dat $\sqrt{pq}$ irrationaal is.
    \end{shortthm}
    Met behulp van \textbf{\cref{lemma1}.} kan dit met behulp van contradictie worden bewezen.
    \begin{proof}[Bewijs van \textbf{Stelling \ref{TheoremA}}] $ $ \newline
        Stel $p,q$ zijn twee priemgetallen. \newline
        Neem aan dat $\sqrt{pq}$ is rationaal.
        In andere woorden $\sqrt{pq}$ is te schrijven als $\sqrt{pq} = \tfrac{a}{b}$ 
        met $a,b \in \mathbb{Z}$ zodat $a$ en $b$ 
        geen gemeenschappelijke deler(s) hebben en $b \neq 0$.

        Dan $pq = \tfrac{a^2}{b^2}$ en $b^2pq = a^2$. \newline
        Omdat $b^2 \in \mathbb{Z}$ geldt dat $p$ deelt $a^2$, 
        en samen met \textbf{\cref{lemma1}.} volgt dat $p$ deelt $a$. \newline 
        Omdat $p$ deelt $a$, is $a$ te schrijven als 
        $a = pk$, met $k$ een willekeurig getal in $\mathbb{Z}$.
        hieruit volgt dat $a^2 = p^2k^2$.

        Wanneer we dit invullen in de vergelijking die we voor $a^2$ hadden gevonden, 
        vinden we dat $b^2pq = p^2k^2$. 
        Nu kan er aan allebei de kanten een $p$ worden weggestreept, 
        om het volgende te vinden: $b^2q = pk^2$. Waaruit blijkt dat $p$ deelt $b^2q$.

        Vanuit \textbf{\cref{lemma1}.} weten we dat dit impliceert dat $p \mid b^2$, 
        of $p \mid q$. Maar omdat $q$ priem is, en $p \neq q$ kan $p$ deelt $q$ niet,
        wat leidt tot het feit dat $p$ deelt $b^2$ 
        en dus (opnieuw volgens \textbf{\cref{lemma1}.}) ook $p$ deelt $b$.

        Maar nu zien we dat $p$ deelt $a$ en $p$ deelt $b$, 
        terwijl de aannamen was dat $a$ en $b$ geen gemeenschappelijke deler(s) hebben.
        We hebben een tegenspraak gevonden met onze orginele aanname, 
        dus er bestaan geen $a$ en $b$ waarvoor $\sqrt{pq} = \tfrac{a}{b}$, 
        met $a$ en $b$ geen gemeenschappelijke deler(s).

        Voor twee priemgetallen $p$ en $q$ geldt $\sqrt{pq}$ is irrationaal. \newline
    \end{proof}
    \newpage
    \item 
    \begin{shortclaim}\label{claim1}
        \textbf{Stelling \ref{TheoremA}.} 
        blijft gelden wanneer de twee getallen $p$ en $q$ niet priem zijn.  
    \end{shortclaim}

    Het tegendeel van deze stelling is te bewijzen door middel van een tegenvoorbeeld.
    \begin{proof}[Bewijs van het tegendeel van \textbf{\cref{claim1}}] $ $ \newline
        Neem twee getallen $p = 2$ en $q = 8$. \newline
        Dan $\sqrt{pq} = \sqrt{16} = 4$. \newline
        4 is niet irrationaal, dus \textbf{\cref{claim1}.} is onwaar. \newline
    \end{proof}

\end{enumerate}



\end{document}