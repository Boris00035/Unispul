\documentclass[12pt, dutch, a4paper]{article}

\usepackage[hidelinks]{hyperref}
\usepackage[tmargin=0.65in,bmargin=0.7in]{geometry}
\usepackage{parskip}
\usepackage{amssymb}
\usepackage{amsmath}
\usepackage[shortlabels]{enumitem}
\usepackage[dutch]{babel}
\selectlanguage{dutch}
\usepackage{amsthm}
\usepackage{cleveref}

\theoremstyle{definition}

\newtheorem{theorem}{Stelling} %\refname{Stelling}??
\newtheorem{lemma}{Lemma}[theorem]
\newtheorem{lemmalos}{Lemma}
\newtheorem{sublemma}{Lemma}[lemma]
\newtheorem{case}{Geval}
\newtheorem{claim}{Claim}

% refname van de theorem veranderen

\newenvironment{shortclaim}
  {\refstepcounter{claim}\textbf{Claim~\theclaim.}}% \begin{shortthm}
{\enskip}
\newenvironment{shortthm}
  {\refstepcounter{theorem}\textbf{Stelling~\thetheorem.}}% \begin{shortthm}
{\enskip}

\begin{document}

\title{Bewijzen - Inleveropgave 4}
\author{B.H.J. van Boxtel}
\date{12 Oktober 2022 - Week 41} 

\maketitle
\pagenumbering{arabic} 

Gegeven is het volgende lemma:
\begin{lemmalos}\label{lemma1}
    Zij $m,n \in \mathbb{Z}$ en $p$ een priemgetal. Als $p \mid mn$, 
    dan geldt dat $p \mid m$ of $\; p \mid n$.
\end{lemmalos}

Ook gegeven is de volgende relatie $R$ op $\mathbb{Z}$, 
voor gehele getallen $a$ en $b$ door de voorwaarde dat $a\,R\,b$ dan en slechts dan als
$b - a$ deelbaar is door zowel $p$ als $q$. \newline
\begin{enumerate}[(a).]
  \item 
  \begin{shortthm}
    $R$ is een equivalentierelatie.
  \end{shortthm}

  Om te laten zien dat $R$ een equivalentierelatie is, 
  moeten we laten zien dat $R$ reflexief, symmetrisch en transitief is.

  \begin{itemize}
    \item 
    \begin{shortclaim} 
      $R$ is reflexief. In andere woorden er geldt $a\,R\,a$.
    \end{shortclaim}
    \begin{proof} $ $ \newline
      We weten dat $q \mid 0$. \newline
      Dus $q \mid a - a$. \newline
      Ook weten we dat $p \mid 0$. \newline
      Dus $p \mid a - a$. \newline
      Dus $a\,R\,a$. \newline
      Dus $R$ is reflexief. \newline
    \end{proof}

    \item 
    \begin{shortclaim} 
      $R$ is symmetrisch. Dat wil zeggen $a\,R\,b$ impliceert $b\,R\,a$. 
    \end{shortclaim}
    \begin{proof} $ $ \newline
      Neem aan $a\,R\,b$. \newline
      Volgens de definitie van $R$ geldt dan $q \mid b - a$. \newline
      Dus $b - a$ is te schrijven als $b - a = kq$ voor een $k \in \mathbb{Z}$. \newline
      Dus $a - b = -kq$. \newline
      Dus $q \mid a - b$.

      Volgens de definitie van $R$ geldt $p \mid b - a$. \newline
      Dus $b - a$ is te schrijven als $b - a = mp$ voor een $m \in \mathbb{Z}$. \newline
      Dus $a - b = -kp$. \newline
      Dus $p \mid a - b$.

      Dus $q \mid a - b$ en $p \mid a - b$, dus $b\,R\,a$. \newline
      We zien dat $b\,R\,a$ volgt uit $a\,R\,b$, dus $R$ is symmetrisch. \newline
    \end{proof}
    \newpage
    \item 
    \begin{shortclaim}
      $R$ is transitief. Dat wil zeggen dat wanneer $a\,R\,b$ en $b\,R\,c$,
      dan $a\,R\,c$.
    \end{shortclaim}

    \begin{proof} $ $ \newline
      Neem aan $a\,R\,b$ en $b\,R\,c$.
      Vanuit de definitie van $R$ volgt dan dat $q$ een deler is van $b - a$,
      en dat $q$ een deler is van $c - b$. \newline
      Dus $b - a$ is te schrijven als $b - a = kq$ met $k \in \mathbb{Z}$. \newline
      En $c - b$ is te schrijven als $c - b = mq$ met $m \in \mathbb{Z}$.
    
      Wanneer we deze twee vergelijkingen bij elkaar optellen, 
      vinden we dat $c - a = q(k + m)$. \newline 
      Omdat $(k + m) \in \mathbb{Z}$, geldt nu dus dat $q \mid c - a$.

      Vanuit de definitie van $R$ volgt ook dat $p$ een deler is van $b - a$,
      en dat $p$ een deler is van $c - b$. \newline
      Dus $b - a$ is te schrijven als $b - a = np$ met $n \in \mathbb{Z}$. \newline
      En $c - b$ is te schrijven als $c - b = lp$ met $l \in \mathbb{Z}$.

      Wanneer we deze twee vergelijkingen bij elkaar optellen, 
      vinden we dat $c - a = p(n + l)$. \newline  
      Omdat $(n + l) \in \mathbb{Z}$, geldt nu dus dat $p \mid c - a$.

      Dus omdat $q \mid c - a$ en $p \mid c - a$, geldt $a\,R\,c$. \newline
      Dus $a\,R\,b$ en $b\,R\,c$ impliceert $a\,R\,c$. \newline 
      Dus $R$ is transitief. \newline
    \end{proof}
  \end{itemize}
  
  \setcounter{claim}{0}
  Dus $R$ is een equivalentierelatie. \newline
  \item 
  \begin{shortthm} \label{theorem2}
    De relatie $a\,R\,b$ geldt dan en slechts dan als $a \equiv b \pmod{pq}$.
  \end{shortthm}

  Dit is een biconditionele implicatie, dus de implicatie moet allebei de kanten op gelden.
  Eerst bewijs ik de implicatie van links naar rechts, 
  dat wil zeggen $a\,R\,b$ impliceert $a \equiv b \pmod{pq}$.

  \begin{proof} $ $ \newline
    Neem aan $a\,R\,b$. \newline
    Dan geldt vanuit de definitie van $R$ dat $p \mid b - a$ en $q \mid b - a$. 

    Omdat $p$ en $q$ priemgetallen zijn, kunnen dit geen delers van elkaar zijn. \newline
    Omdat geldt $p \mid b - a$ \'en $q \mid b - a$, 
    kunnen we $b - a$ schrijven als $b - a = kpq$ voor een $k \in \mathbb{Z}$.

    Dus $b - a \equiv 0 \pmod{pq}$ en $b \equiv a \pmod{pq}$. \newline
  \end{proof}

  Om het bewijs van \textbf{Stelling \ref{theorem2}.} af te maken, 
  bewijzen we nu de implicatie de andere kant op. 
  Dat wil zeggen $a \equiv b \pmod{pq}$ impliceert $a\,R\,b$.
  
  \begin{proof} $ $ \newline
    Neem aan $a \equiv b \pmod{pq}$. \newline
    Dus $a - b \equiv 0 \pmod{pq}$. \newline
    Dus $b - a \equiv 0 \pmod{pq}$. \newline
    Dus $pq \mid b - a$. \newline  
    Dan kunnen we $  - a$ schrijven als $b - a = kpq$ voor een $k \in \mathbb{Z}$. \newline
    Dus $p \mid b - a$ en $q \mid b - a$. \newline
  \end{proof}

  \item 
  \begin{shortthm}
    De verzameling van equivalentieklassen van $R$ is gelijk aan $\mathbb{Z}_{pq}$.
  \end{shortthm}

  \begin{proof} $ $ \newline
    Volgens \textbf{Stelling \ref{theorem2}.} geldt $a\,R\,b$ dan en slechts dan als 
    $a \equiv b \pmod{pq}$. 
    Dit betekent dat een getal $b$ alleen equivalent kan zijn aan $a$ dan en slechts dan 
    als $b$ congruent is aan $a \pmod{pq}$.
    Dus alle getallen in de equivalentieklassen van $R$ met representant $a$,
    zitten ook in de equivalentieklassen van de getallen $\pmod{pq}$ met representant $a$. 
    Ook geldt dat  
    alle getallen in de equivalentieklassen van de getallen $\pmod{pq}$ met representant $a$
    in de equivalentieklassen van $R$ met representant $a$ zitten. 
    Dus de verzameling van equivalentieklassen van $R$ is gelijk aan $\mathbb{Z}_{pq}$.  
    \newline
  \end{proof}
\end{enumerate}



\end{document}