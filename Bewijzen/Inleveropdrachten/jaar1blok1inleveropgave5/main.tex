\documentclass[12pt, dutch, a4paper]{article}

\usepackage[hidelinks]{hyperref}
\usepackage[tmargin=0.4in,bmargin=0.77in]{geometry}
\usepackage{parskip}
\usepackage{amssymb}
\usepackage{amsmath}
\usepackage[shortlabels]{enumitem}
\usepackage[dutch]{babel}
\selectlanguage{dutch}
\usepackage{amsthm}

\theoremstyle{definition}

\newtheorem{theorem}{Stelling}
\newtheorem{lemma}{Lemma}[theorem]
\newtheorem{lemmalos}{Lemma}
\newtheorem{sublemma}{Lemma}[lemma]
\newtheorem{case}{Geval}
\newtheorem{claim}{Claim}

\newenvironment{shortclaim}
  {\refstepcounter{claim}\textbf{Claim~\theclaim.}}% \begin{shortthm}
{\enskip}
\newenvironment{shortthm}
  {\refstepcounter{theorem}\textbf{Stelling~\thetheorem.}}% \begin{shortthm}
{\enskip}

\begin{document}

\title{Bewijzen - Inleveropgave 5}
\author{B.H.J. van Boxtel}
\date{19 Oktober 2022 - Week 42} 

\maketitle
\pagenumbering{arabic} 

We bekijken een spel dat wordt gespeeld door twee spelers, Alice en Bob. 
Zij kiezen een natuurlijk getal $n$ en maken dan een stapel met $n$ muntjes. 
Omstebeurt mogen ze 1, 2 of 3 muntjes van deze stapel pakken. 
Degene die het laatste muntje pakt verliest. Alice begint. 
Stel dat $n \equiv 1 \pmod{4}$. We bewijzen dat Bob altijd kan winnen, wat Alice ook doet.

We weten dat $n \equiv 1 \pmod{4}$, 
dus $n$ is te schrijven als $n = 4k + 1$ met $k \in \mathbb{Z}_{\geq 0}$.  
We bewijzen nu dat Alice verliest voor elke $n$ van deze vorm,
door met inductie te bewijzen dat Alice verliest voor elke $k$.  

\begin{proof} $ $ \newline
  Bij $k = 0$ is de stapel van de vorm $n = 1$. 
  Haar enige optie is om het laatste muntje te pakken, dus verliest ze.

  Neem aan dat Alice verliest als zij 
  en aan de beurt is, en de stapel van de vorm $n = 4k + 1$ is.

  Beschouw nu het geval dat $n = 4(k + 1) + 1 = 4k + 5$.
  Gegeven is dat Alice begint. We kunnen nu 3 gevallen onderscheiden, 
  voor elke keuze die Alice kan maken.
  \begin{case}
    Alice pakt 1 muntje van de stapel. 
    De stapel is nu van de vorm $n = 4k + 4$. 
    Bob kan nu 3 muntjes van de stapel halen. De stapel is nu van de vorm $n = 4k + 1$. 
    Alice is nu aan de beurt, en de stapel is van de vorm $n = 4k + 1$.
    Vanuit de aanname weten we dat ze nu verliest.
  \end{case}

  \begin{case}
    Alice pakt 2 muntjes van de stapel.
    De stapel is nu van de vorm $n = 4k + 3$. 
    Bob kan nu 2 muntjes van de stapel halen. De stapel is nu van de vorm $n = 4k + 1$.
    Alice is nu aan de beurt, en de stapel is van de vorm $n = 4k + 1$.
    Vanuit de aanname weten we dat ze nu verliest.
  \end{case}

  \begin{case}
    Alice pakt 3 muntjes van de stapel.
    De stapel is nu van de vorm $n = 4k + 2$. 
    Bob kan nu 1 muntje van de stapel halen. De stapel is nu van de vorm $n = 4k + 1$.
    Alice is nu aan de beurt, en de stapel is van de vorm $n = 4k + 1$.
    Vanuit de aanname weten we dat ze nu verliest.
  \end{case}

  Dus het feit dat Alice verliest als ze 
  en aan de beurt is, en de stapel van de vorm $n = 4k + 1$ is,
  impliceert dat Alice ook verliest als ze 
  en aan de beurt is, en de stapel van de vorm $n = 4k + 5$ is.

  Samen met het feit dat Alice verliest in het geval dat ze en aan de beurt is, en $k = 0$, 
  betekent dit dat Alice verliest voor alle $k$, 
  en dus voor alle stapels waarvoor geldt $n \equiv 1 \pmod{4}$. 

\end{proof} 






\end{document}