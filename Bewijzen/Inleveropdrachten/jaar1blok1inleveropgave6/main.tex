\documentclass[12pt, dutch, a4paper]{article}

\usepackage[hidelinks]{hyperref}
\usepackage[tmargin=0.8in, bmargin=1in]{geometry}
\usepackage{parskip}
\usepackage{amssymb}
\usepackage{amsmath}
\usepackage[shortlabels]{enumitem}
\usepackage[dutch]{babel}
\selectlanguage{dutch}
\usepackage{amsthm}
\usepackage{cleveref}

\theoremstyle{definition}

\newtheorem{theorem}{Stelling}
\newtheorem{lemma}{Lemma}[theorem]
\newtheorem{lemmalos}{Lemma}
\newtheorem{sublemma}{Lemma}[lemma]
\newtheorem{case}{Geval}
\newtheorem{claim}{Claim}

\newenvironment{shortclaim}
  {\refstepcounter{claim}\textbf{Claim~\theclaim.}}% \begin{shortthm}
{\enskip}
\newenvironment{shortthm}
  {\refstepcounter{theorem}\textbf{Stelling~\thetheorem.}}% \begin{shortthm}
{\enskip}

\title{Bewijzen - Inleveropgave 6}
\author{B.H.J. van Boxtel}
\date{25 Oktober 2022 - Week 43} 

\begin{document}

\maketitle  
\pagenumbering{arabic} 

Gegeven is de functie 
$f: \mathbb{R} \longrightarrow \mathbb{R}, \: \: x \mapsto 3x^5$.
Een functie $g(x)$ is continu in $a$ als $\lim_{x\to a} g(x) = g(a)$.

\begin{theorem}
  Zij $a \geq 0$ willekeurig maar vast, dan is $f$ continu in $a$.
\end{theorem}

In andere woorden, er moet worden bewezen dat $\lim_{x\to a} f(x) = f(a)$, 
oftewel $\lim_{x\to a}3x^5 = 3a^5$.

Dit is te bewijzen doormiddel van een $\epsilon-\delta$-bewijs.
\begin{proof} $ $ \newline
  Kies \[\delta = 
  \min{\left\{1,\frac{\epsilon}{15a^4 + 30a^4 + 30a^2 + 15a + 3}\right\}}\]

  Neem nu aan:
  \begin{equation} \label{start}
    \lvert x - a \rvert < \delta 
  \end{equation}
  We willen laten zien dat hieruit volgt dat:
  \begin{equation} \label{goal}
    \lvert 3x^5 - 3a^5 \rvert < \epsilon
  \end{equation}
  Door onze keuze van $\delta$, geldt het volgende:
  \begin{equation}
    \delta \leq 1 
  \end{equation}
  En hierdoor geldt ook:
  \begin{equation} \label{xbound}
    x \leq a + 1
  \end{equation}
  We zien dat we het linkerdeel van \cref{goal} kunnen herschrijven als:
  \begin{equation} \label{buitenhaken}
    \lvert 3x^5 - 3a^5 \rvert = 
    \lvert x - a \rvert \lvert 3x^4 + 3x^3a + 3x^2a^2 + 3xa^3 + 3a^4 \rvert 
  \end{equation}
  Door de driehoeks ongelijkheid kunnen we schrijven:
  \begin{equation}
    \lvert 3x^4 + 3x^3a + 3x^2a^2 + 3xa^3 + 3a^4 \rvert \leq
    \lvert 3x^4\rvert + \lvert3x^3a\rvert + \lvert3x^2a^2\rvert + 
    \lvert3xa^3\rvert + \lvert3a^4 \rvert
  \end{equation}
  Door \cref{xbound} geldt het volgende:
  \begin{multline}
    \lvert 3x^4 + 3x^3a + 3x^2a^2 + 3xa^3 + 3a^4 \rvert \\ \leq 
    \lvert 3(a + 1)^4\rvert + \lvert3(a + 1)^3a\rvert + \lvert3(a + 1)^2a^2\rvert + 
    \lvert3(a + 1)a^3\rvert + \lvert3a^4 \rvert
  \end{multline}
  Gegeven is dat $a \geq 0$, 
  dus we kunnen aan de rechterkant de absoluutstrepen weglaten, 
  en vervolgens de hele vergelijking simplificeren tot:
  \begin{equation}
    \lvert 3x^4 + 3x^3a + 3x^2a^2 + 3xa^3 + 3a^4 \rvert \leq
    15a^4 + 30a^3 + 30a^2 + 15a + 3
  \end{equation}
  Wanneer we dit samennemen met \cref{start} vinden we:
  \begin{equation}
    \lvert x - a \rvert 
    \lvert 3x^4 + 3x^3a + 3x^2a^2 + 3xa^3 + 3a^4 \rvert \leq
    \delta(15a^4 + 30a^3 + 30a^2 + 15a + 3)
  \end{equation}
  Door \cref{buitenhaken} 
  kunnen we het linkerdeel van de ongelijheid als volgt schrijven:
  \begin{equation} \label{bijnaklaar}
    \lvert 3x^5 - 3a^5 \rvert \leq \delta(15a^4 + 30a^3 + 30a^2 + 15a + 3)
  \end{equation}
  Door onze keuze van $\delta$ weten we dat:
  \begin{equation}
    \delta \leq \frac{\epsilon}{15a^4 + 30a^4 + 30a^2 + 15a + 3}
  \end{equation}
  Hierdoor kunnen we \cref{bijnaklaar} ook schrijven als:
  \begin{equation}
    \lvert 3x^5 - 3a^5 \rvert \leq 
    \frac{\epsilon (15a^4 + 30a^4 + 30a^2 + 15a + 3)}
    {15a^4 + 30a^4 + 30a^2 + 15a + 3}
  \end{equation}
  Dus:
  \begin{equation}
    \lvert 3x^5 - 3a^5 \rvert \leq \epsilon 
  \end{equation}

\end{proof}
\end{document}