\documentclass[12pt, a4paper]{article}
\usepackage[hidelinks]{hyperref}
\usepackage[tmargin=0.8in, bmargin=1in,]{geometry}
\usepackage{parskip}
\usepackage{amssymb}
\usepackage{amsmath}
\usepackage[shortlabels]{enumitem}
\usepackage[dutch]{babel}
\selectlanguage{dutch}
\usepackage{pgfplots}
\pgfplotsset{width=7.5cm, compat=1.18}
\usepackage{amsthm}
\usepackage{cleveref}

\title{Inleveropgave 2 Linalg}
\author{Boris van Boxtel en Brechtje Poppen}
\date{27 Oktober 2022}

\begin{document}

\maketitle 

\begin{enumerate}[(a.)]
\item  \underline{Stelling:} 

Laat $U$ en $V$ twee lineaire deelruimten zijn van $\mathbf{R}$$^n$. Dan geldt dat $U + V$ = \{$\vec u$ + $\vec v$ $\vert$ $\vec u$ $\in$ $U$, $\vec v$ $\in$ $V$\} ook een lineaire deelruimte is van $\mathbf{R}$$^n$.

\underline{Bewijs:}\begin{proof}[\unskip\nopunct]
Er moeten drie eigenschappen bewezen worden, namelijk:
\begin{enumerate}[1.]
\item $\vec{0}$ $\in$ $U + V$
\item Als $\vec{x}$, $\vec{y}$ $\in$ $U + V$, dan geldt ook $\vec{x}$ + $\vec{y}$ $\in$ $U + V$.
\item Als $\vec{x}$ $\in$ $U + V$ en $\lambda$ $\in$ $\mathbf{R}$ dan $\lambda$ $\cdot$ $\vec{x}$ $\in$ $U + V$. 
\end{enumerate}
Belangrijk om op te merken is dat voor $\vec{a}$ $\in$ $U$ en $\vec{b}$ $\in$ $V$ en $\vec{a}$ + $\vec{b}$ = $\vec{c}$ geldt dat $\vec{c}$ $\in$ $U + V$. 

We bewijzen eerst de eerste eigenschap. 

Gegeven was dat $U$ en $V$ beide deelruimtes zijn van $\mathbf{R}$$^n$. Uit de definitie van een deelruimte volgt dan dat $\vec{0}$ $\in$ $U$ en $\vec{0}$ $\in$ $V$. Er geldt dat $\vec{0}$ = $\vec{0}$ + $\vec{0}$, dus als $\vec{0}$ $\in$ $U$ en $\vec{0}$ $\in$ $V$, dan geldt ook dat $\vec{0}$ $\in$ $U + V$. 

Nu kijken we naar de tweede eigenschap.

Neem een $\vec{x}$ $\in$ $U + V$. Er geldt dan dat $\vec{x}$ = $\vec{x_{u}}$ + $\vec{x_{v}}$ met $\vec{x_{u}}$ $\in$ $U$ en $\vec{x_{v}}$ $\in$ $V$. Neem ook een $\vec{y}$ $\in$ $U + V$, waarvoor dan geldt dat $\vec{y}$ = $\vec{y_{u}}$ + $\vec{y_{v}}$ met $\vec{y_{u}}$ $\in$ $U$ en $\vec{y_{v}}$ $\in$ $V$. Additie van $\vec{x}$ en $\vec{y}$ geeft $\vec{x}$ + $\vec{y}$ = ($\vec{x_{u}}$ + $\vec{x_{v}}$) + ($\vec{y_{u}}$ + $\vec{y_{v}}$). Dit is ook te schrijven als $\vec{x}$ + $\vec{y}$ = ($\vec{x_{u}}$ + $\vec{y_{u}}$) + ($\vec{u_{v}}$ + $\vec{y_{v}}$). Hierbij geldt dat $\vec{x_{u}}$ $\in$ $U$ en $\vec{y_{u}}$ $\in$ $U$, en aangezien $U$ een deelruimte van $\mathbf{R}$$^n$ is geldt $\vec{x_{u}}$ + $\vec{y_{u}}$ $\in$ $U$. Een soortgelijk argument leidt tot $\vec{x_{v}}$ + $\vec{y_{v}}$ $\in$ $V$. Uit $\vec{x_{u}}$ + $\vec{y_{u}}$ $\in$ $U$ en $\vec{x_{v}}$ + $\vec{y_{v}}$ $\in$ $V$ volgt dat ($\vec{x_{u}}$ + $\vec{y_{u}}$) + ($\vec{u_{v}}$ + $\vec{y_{v}}$) $\in$ $U + V$, dus als $\vec{x}$, $\vec{y}$ $\in$ $U + V$, dan geldt ook $\vec{x}$ + $\vec{y}$ $\in$ $U + V$.

Ten slotte bewijzen we nog dat de laatste eigenschap geldig is. 

Neem een $\vec{x}$ $\in$ $U + V$ en een $\lambda$ $\in$ $\mathbf{R}$. Er geldt voor $\vec{x}$ dat $\vec{x}$ = $\vec{x_{u}}$ + $\vec{x_{v}}$ met $\vec{x_{u}}$ $\in$ $U$ en $\vec{x_{v}}$ $\in$ $V$. Als we bij de gelijkheid $\vec{x}$ = $\vec{x_{u}}$ + $\vec{x_{v}}$ een scalaire vermenigvuldiging met $\lambda$ uitvoeren krijgen we $\lambda$ $\cdot$ $\vec{x}$ = $\lambda$ $\cdot$ ($\vec{x_{u}}$ + $\vec{x_{v}}$). Dit is ook te schrijven als $\lambda$ $\cdot$ $\vec{x}$ = $\lambda$ $\cdot$ $\vec{x_{u}}$ + $\lambda$ $\cdot$ $\vec{x_{v}}$. Aangezien $U$ een deelruimte is van $\mathbf{R}$$^n$ en er geldt dat $\vec{x_{u}}$ $\in$ $U$, geldt ook dat $\lambda$ $\cdot$ $\vec{x_{u}}$ $\in$ $U$. Hetzelfde argument leidt tot $\lambda$ $\cdot$ $\vec{x_{v}}$ $\in$ $V$. Uit $\lambda$ $\cdot$ $\vec{x_{u}}$ $\in$ $U$ en $\lambda$ $\cdot$ $\vec{x_{v}}$ $\in$ $V$ volgt dat $\lambda$ $\cdot$ $\vec{x}$ $\in$ $U + V$, dus als $\vec{x}$ $\in$ $U + V$ en $\lambda$ $\in$ $\mathbf{R}$ dan geldt ook dat $\lambda$ $\cdot$ $\vec{x}$ $\in$ $U + V$. 

Er is nu dus bewezen dat elk van de drie eigenschappen geldt, dus is $U + V$ een lineaire deelruimte van $\mathbf{R}$$^n$.\end{proof} 

\item \label{b} Gegeven is de matrix $A$:
\begin{equation} 
    A =
    \begin{pmatrix}
        1 & 1 & 1 & 2 & -1\\
        0 & 2 & 0 & 0 & 4\\
        0 & 0 & 0 & 2 & 2\\
    \end{pmatrix}
\end{equation}
 
Laat $U$ de lineaire deelruimte van $\mathbf{R}$$^3$ zijn opgespannen door de eerste drie kolommen van $A$. Dus het opspansel van $U$ is gegeven door:
\begin{equation} 
    \text{Span}(U) = 
    \text{Span}\left\{
        \begin{pmatrix}
            1 \\
            0 \\
            0 \\
        \end{pmatrix}
        ,
        \begin{pmatrix}
            1 \\
            2 \\
            0 \\
        \end{pmatrix}
        ,
        \begin{pmatrix}
            1 \\
            0 \\
            0 \\
        \end{pmatrix}
    \right\}
\end{equation}
De basis van een deelruimte wordt gegeven door de onafhankelijke vectoren in het opspansel. 
Om deze onafhankelijke vectoren te vinden, 
schrijven we deze vectoren als volgt in een matrix vergelijking:
\begin{equation}
    \left(\begin{array}{ccc|c}
        1 & 1 & 1 & u_1\\
        0 & 2 & 0 & u_2\\
        0 & 0 & 0 & u_3\\
    \end{array}\right)
\end{equation}
Waar $u_1,u_2$ en $u_3$ de elementen zijn van een vector $\vec{u}$ uit $U$.
We zien dat deze matrix al in rijgereduceerde vorm is, 
en twee pivot elementen in de eerste twee kolommen heeft.
Dus we kunnen als basis voor $U$ hetvolgende nemen:
\begin{equation}
    \text{Basis}(U) = \left\{
        \begin{pmatrix}
            1 \\
            0 \\
            0 \\
        \end{pmatrix}
        ,
        \begin{pmatrix}
            1 \\
            2 \\
            0 \\
        \end{pmatrix}
    \right\}
\end{equation}
We hebben twee vectoren in de basis van $U$, dus de dimensie is $U$.

\item
Het opspansel van $V$ wordt gegeven door:
\begin{equation}
    \text{Span}(V) = \text{Span}\left\{
        \begin{pmatrix}
            2 \\
            0 \\
            2 \\
        \end{pmatrix}
        ,
        \begin{pmatrix}
            -1 \\
            4 \\
            2 \\
        \end{pmatrix}
    \right\}
\end{equation}
Dus we kunnen op dezelfde manier als bij \ref{b} de volgende matrixvergelijking opstellen:
\begin{equation}
    \left(\begin{array}{cc|c}
        2 & -1 & v_1\\
        0 & 4 & v_2\\
        2 & 2 & v_3\\
    \end{array}\right)
\end{equation}
Waar $v_1, v_2$ en $v_3$ elementen van de vector $\vec{v}$ uit $V$. 
Deze matrix is nog niet in rij gereduceerde vorm, 
dus als we deze matrix vegen, vinden we het volgende:  
\begin{equation}
    \left(\begin{array}{cc|c}
        2 & -1 & v_1\\
        0 & 4 & v_2\\
        0 & 0 & v_3 - v_1 - \frac{3}{4}v_2\\
    \end{array}\right)
\end{equation}
En zien we dat deze matrix twee pivot elementen in de eerste twee kolommen heeft, 
en dus de basis van $V$ is:
\begin{equation}
    \text{Basis}(V) = \left\{
        \begin{pmatrix}
            2 \\
            0 \\
            2 \\
        \end{pmatrix}
        ,
        \begin{pmatrix}
            -1 \\
            4 \\
            2 \\
        \end{pmatrix}
    \right\}
\end{equation}
\item  
De dimensie van $U \cap V$ 
kan niet groter zijn dan de dimensie van de deelverzameling met de laagste dimensie.
$V$ en $U$ hebben allebei dimensie 2, dus de maximale dimensie van $U \cap V$ is 2.

\item 
Gevraagd is om de basis van $U$ $\cap$ $V$ te bepalen. Dit doen we door alle vectoren $\vec{x}$ waarvoor geldt $\vec{x}$ $\in$ $U$ $\cap$ $V$ te bepalen. Uit $\vec{x}$ $\in$ $U$ $\cap$ $V$ volgt dat $\vec{x}$ te schrijven is als een lineaire combinatie van de vectoren uit de basis van zowel $U$ als $V$, dus er geldt: 
\begin{equation}
    \vec{x} = 
    \lambda_1
    \begin{pmatrix}
        1\\
        0\\
        0
    \end{pmatrix} 
    + \lambda_2
    \begin{pmatrix}
        1\\
        2\\
        0
    \end{pmatrix}    
\end{equation}
met $\lambda_1$, $\lambda_2$ $\in$ $\mathbf{R}$, en er geldt:
\begin{equation}
    \vec{x} = \mu_1
    \begin{pmatrix}
        2\\
        0\\
        2
    \end{pmatrix} 
    + \mu_2
    \begin{pmatrix}
        -1\\
        4\\
        2
    \end{pmatrix} 
\end{equation}
met $\mu_1$, $\mu_2$ $\in$ $\mathbf{R}$. Hieruit volgt het volgende:
\begin{equation}
    \lambda_1
    \begin{pmatrix}
        1\\
        0\\
        0
    \end{pmatrix} 
    + \lambda_2
    \begin{pmatrix}
        1\\
        2\\
        0
    \end{pmatrix}
    = 
    \mu_1
    \begin{pmatrix}
        2\\
        0\\
        2
    \end{pmatrix} 
    + \mu_2
    \begin{pmatrix}
        -1\\
        4\\
        2
    \end{pmatrix} 
\end{equation}

Dit stelsel vergelijkingen lossen we nu op voor de coëfficiënten. Dit wordt gedaan door een matrix $B$ op te stellen met de volgende vectoren als kolommen:
\begin{equation}
    \begin{pmatrix}
        1\\
        0\\
        0
    \end{pmatrix}
    , 
    \begin{pmatrix}
        1\\
        2\\
        0
    \end{pmatrix}
    ,
    \begin{pmatrix}
        2\\
        0\\
        2
    \end{pmatrix}
    ,
    \begin{pmatrix}
        -1\\
        4\\
        2
    \end{pmatrix}   
\end{equation}
En daarna lossen we het bijbehorende homogene stelsel vergelijkingen op door middel van rijreductie. We krijgen het volgende stelsel:
\begin{equation}
    \left(\begin{array}{cccc|c}
        1 & 1 & 2 & -1 & 0\\
        0 & 2 & 0 & 4 & 0\\
        0 & 0 & 2 & 2 & 0\\
    \end{array}\right)
\end{equation}
Een oplossing voor dit stelsel is $(5,-2,1,-1)^t$, dus de basis van $U \cap V$ is:
\begin{equation}
    \text{Basis}(U \cap V) = 
    5
    \begin{pmatrix}
        1 \\ 
        0 \\ 
        0 \\
    \end{pmatrix}
    - 2
    \begin{pmatrix}
        1 \\
        2 \\
        0 \\
    \end{pmatrix}
    =
    \begin{pmatrix}
        3 \\
        -4 \\
        0 \\
    \end{pmatrix}
\end{equation} 

\item 
We weten vanuit \textbf{Stelling 4.4.1} van het dictaat dat geldt:
\begin{equation}
    \text{dim}(U + V) = \text{dim}(U) + \text{dim}(V) - \text{dim}(U \cap V) 
\end{equation}
Dus als we dit invullen krijgen we:
\begin{equation}
    \text{dim}(U + V) = 2 + 2 - 1 = 3
\end{equation}

\item
We weten dat we een vector in een deelruimte als een lineare combinatie van basisvectoren 
van deze deelruimte kunnen schrijven, 
en als we dit doen voor een $\vec{u} \in U$ en een $\vec{v} \in V$, krijgen we het volgende:
\begin{align}
    \vec{u} &=  
    \lambda_1\begin{pmatrix}
        1 \\
        0 \\
        0 \\
    \end{pmatrix}
    +
    \lambda_2\begin{pmatrix}
        1 \\
        2 \\
        0 \\
    \end{pmatrix} \\
    \vec{v} &=
    \mu_1\begin{pmatrix}
        2 \\
        0 \\
        2 \\
    \end{pmatrix}
    +
    \mu_2\begin{pmatrix}
        -1 \\
        4 \\
        2 \\
    \end{pmatrix}
\end{align} 
Als we deze vergelijkingen optellen, krijgen we:
\begin{equation}
    \lambda_1\begin{pmatrix}
        1 \\
        0 \\
        0 \\
    \end{pmatrix}
    +
    \lambda_2\begin{pmatrix}
        1 \\
        2 \\
        0 \\
    \end{pmatrix}
    +
    \mu_1\begin{pmatrix}
        2 \\
        0 \\
        2 \\
    \end{pmatrix}
    +
    \mu_2\begin{pmatrix}
        -1 \\
        4 \\
        2 \\
    \end{pmatrix}
    =
    \vec{u} + \vec{v}
\end{equation}
Wat precies de definitie is voor een vector $\vec{x}$ uit de somruimte $U + V$. 
Dit kunnen we als volgt als matrixvergelijking schrijven:
\begin{equation}
    \left(\begin{array}{cccc|c}
        1 & 1 & 2 & -1 & x_1\\
        0 & 2 & 0 & 4 & x_2\\
        0 & 0 & 2 & 2 & x_3\\
    \end{array}\right)
\end{equation}
Deze vergelijking is al in rij gereduceerde vorm, 
dus een basis is gegeven door de kolommen met de pivots:
\begin{equation}
    \text{Basis}(U + V) = 
    \left\{ 
        \begin{pmatrix}
            1 \\
            0 \\
            0 \\
        \end{pmatrix}
        ,
        \begin{pmatrix}
            1 \\
            2 \\
            0 \\
        \end{pmatrix}
        ,
        \begin{pmatrix}
            2 \\
            0 \\
            2 \\
        \end{pmatrix}
    \right\}
\end{equation}
En ook het volgende is een basis voor $U + V$:
\begin{equation}
    \text{Basis}(U + V) =
    \left\{ 
        \begin{pmatrix}
            2 \\
            0 \\
            0 \\
        \end{pmatrix}
        ,
        \begin{pmatrix}
            1 \\
            2 \\
            0 \\
        \end{pmatrix}
        ,
        \begin{pmatrix}
            2 \\
            0 \\
            2 \\
        \end{pmatrix}
    \right\}
\end{equation}
\end{enumerate}
\end{document}