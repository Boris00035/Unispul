\documentclass[12pt, a4paper]{article}
\usepackage[hidelinks]{hyperref}
\usepackage[tmargin=0.8in, bmargin=1in]{geometry}
\usepackage{parskip}
\usepackage{amssymb}
\usepackage{amsmath}
\usepackage[shortlabels]{enumitem}
\usepackage[dutch]{babel}
\selectlanguage{dutch}
\usepackage{pgfplots}
\pgfplotsset{width=7.5cm, compat=1.18}
\usepackage{amsthm}
\usepackage{cleveref}

\title{Inlever 1 Lial 2}
\author{Boris van Boxtel en Lotte Gritter}
\date{November 2022}

\begin{document}

\maketitle 

\begin{enumerate}[(a).]
    \item \label{a}
        Lemma 7.3.2. De kern van een lineaire afbeelding $A: V \rightarrow W$ is een lineaire deelruimte van $V$. \newline 
        De kern van $A$ is de verzameling van alle elementen in $A$ waarvoor geldt dat $A(\mathbf{x})=0$. We gaan bewijzen dat ker$(A)$ een lineaire deelruimte van V is door te laten zien dat ker$(A)$ niet leeg is, en dat optelling en scalaire vermenigvuldiging gedefiniëerd zijn.

        \begin{proof}\leavevmode
            \begin{enumerate}[1.]
                \item De kern van $A$ bevat de nulvector. 
                
                \item Stel $\mathbf{x},\mathbf{y}$ in ker$(A)$. Dan geldt dat $A(\mathbf{x}+\mathbf{y})=A(\mathbf{x})+A(\mathbf{y})= \mathbf{0} + \mathbf{0}=\mathbf{0}$. Dus $x+y$ zit in ker$(A)$. 
                
                \item Stel $\lambda$ in $\mathbb{R}$ en $\mathbf{x}$ in ker$(A)$. Vanwege lineairiteit geldt dat $A(\lambda \mathbf{x})=\lambda A(\mathbf{x}) = \lambda \mathbf{0} = \mathbf{0}$.
            \end{enumerate}
        \end{proof}

    \item \label{b}
        Lemma 7.3.5. Zij $V, W$ een tweetal vectorruimten en $A: V \rightarrow W$ een lineaire afbeelding. Dan is $A(V)$ een lineaire deelruimte van $W$.
        \newline Het bewijs hiervoor lijkt op het bewijs bij \ref{a}

        \begin{proof}\leavevmode
            \begin{enumerate}[1.]
            \item $A(V)$ is niet leeg, want $\mathbf{0}=A(0) \in A(V)$.
            
            \item
            \end{enumerate} 
        \end{proof}

    \item \label{c}
        Laat $V$ een vectorruimte zijn van dimensie $n$ en $B$ een basis van $V$. (hier moet nog tekst). \newline
        We gaan bewijzen dat $f_B$ een isomorfisme geeft tussen $V$ en $\mathbb{R}_n$, in andere woorden dat $f_B$ een bijectieve lineare functie is.

        het bewijs hiervoor bestaat uit drie delen, het eerste dat de functie linear is, het tweede het bewijs dat $f_B$ injectief is, en het derde dat $f_B$ surjectief is.

        \newpage
        \begin{proof}\leavevmode
            \begin{enumerate}[1.]
                \item 
                Neem aan: 
                \begin{equation} \label{verg1}
                    f_B(\mathbf{x}) = f_B(\mathbf{y})
                \end{equation}
                
                met $\mathbf{x}, \mathbf{y} \in V$. Gegeven is dat $B$ een basis is van $V$. Met de definitie van $f_B$ gegeven in het dictaat, kunnen we dit ook schrijven als:
                % \begin{equation}
                    
                % \end{equation}

            \end{enumerate}
        \end{proof}

        % \begin{proof}
        % Als $f_B$ isomorfisme geeft tussen $V$ en $\mathbb{R}n$, dan is $f_B: V \rightarrow \mathbb{R}_n$ een bijectieve lineaire afbeelding. Als $f_B: V \rightarrow \mathbb{R}_n$ een bijectieve lineaire afbeelding is betekent dat dat de omgekeerde afbeelding $f_B^{-1} \mathbb{R}_n \rightarrow V$ ook een lineaire afbeelding is (stelling 7.3.4, bewezen in het dictaat).
        % % Omdat het eigenlijk alleen een verandering van basis is en de nieuwe coördinaten/lineaire combinatie die dan ontstaat per x in V uniek is, en de omgekeerde afbeelding ook een verandering van basis is is het bijectief, maar dat moet ik nog netjes in wiskundigere termen uitschrijven.
        % \end{proof}

 
        \item \label{d}
    
\begin{proof}
   
\end{proof}

\end{enumerate}
\end{document}