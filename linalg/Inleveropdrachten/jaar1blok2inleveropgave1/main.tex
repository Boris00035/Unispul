\documentclass[12pt, a4paper]{article}
\usepackage[hidelinks]{hyperref}
\usepackage[tmargin=0.8in, bmargin=1in,]{geometry}
\usepackage{parskip}
\usepackage{amssymb}
\usepackage{amsmath}
\usepackage[shortlabels]{enumitem}
\usepackage[dutch]{babel}
\selectlanguage{dutch}
\usepackage{pgfplots}
\pgfplotsset{width=7.5cm, compat=1.18}
\usepackage{amsthm}
\usepackage{cleveref}

\title{Inlever 1 Lial 2}
\author{Boris van Boxtel en Lotte Gritter}
\date{November 2022}

\begin{document}

\maketitle 

\begin{enumerate}[(a).]
    \item \label{a}
    Lemma 7.3.2. De kern van een lineaire afbeelding $A: V \rightarrow W$ is een lineaire deelruimte van $V$. \newline 
    De kern van $A$ is de verzameling van alle elementen in $A$ waarvoor geldt dat $A(x)=0$. We gaan bewijzen dat ker$(A)$ een lineaire deelruimte van $V$ is door te laten zien dat de nulvector in de kern van $A$ zit, en dat de kern van $A$ gesloten is onder optelling en scalaire vermenigvuldiging.

        \begin{proof}\leavevmode
            \begin{enumerate}[1.]
                \item De kern van $A$ bevat de nulvector omdat $A(\mathbf{0_v})=\mathbf{0_v}$.
   % Dit is vanzelfsprekend en ik weet niet zo goed hoe ik dat zou moeten bewijzen...             
                \item Stel dat $x,y$ elementen zijn in ker$(A)$. Dan geldt dat $A(x+y)=A(x)+A(y)= \mathbf{0} + \mathbf{0}=\mathbf{0}$. Dus $x+y$ zit in ker$(A)$. 
                
                \item Stel $\lambda$ in $\mathbb{R}$ en $x$ in ker$(A)$. A is een lineaire afbeelding, dus vanwege lineairiteit geldt dat $A(\lambda x)=\lambda A(x) = \lambda \mathbf{0} = \mathbf{0}$.
            \end{enumerate}
        \end{proof}

    \item \label{b}
    Lemma 7.3.5. Zij $V, W$ een tweetal vectorruimten en $A: V \rightarrow W$ een lineaire afbeelding. Dan is $A(V)$ een lineaire deelruimte van $W$.
    \newline Het bewijs hiervoor lijkt op het bewijs bij \ref{a}

        \begin{proof}\leavevmode
          \begin{enumerate}[1.]
            \item Stel $\mathbf{x}$ en $\mathbf{y}$ zijn elementen in $V$. Dan is $\mathbf{x+y}$ ook een element in $V$. Dus $x+y=A(x)+A(y)=A(x+y)$, en $A(V)$ is dus gesloten onder optelling.

            \item Stel $\lambda$ is een reëel getal en $\mathbf{x}$ is een element in $V$. Dan is $\lambda \mathbf{x}$ ook een element in $V$. Er bestaat een element $y$ waarvoor geldt dat $A(x)=y$. Hieruit volgt dat $\lambda \mathbf{y}=\lambda A(x)= A(\lambda x)$. De lineaire afbeelding is dus gesloten onder scalaire vermenigvuldiging.

            \item Stel $\mathbf{0_v}$ is de nulvector in $V$. We weten dat $\mathbf{0_v}+\mathbf{0_v}=\mathbf{0_v}$. Hieruit, en uit lineairiteit, volgt dat 
            \begin{equation}
                A(\mathbf{0_v})=A(\mathbf{0_v}+\mathbf{0_v})=A(\mathbf{0_v})+A(\mathbf{0_v})
            \end{equation} \newline
            We trekken nu aan beide kanten van de vergelijking $A(\mathbf{0_v})$ af, dit geeft: 
            \begin{equation}
                \mathbf{0_w}=A(\mathbf{0_v})
            \end{equation}
            Dus A(V) bevat de nulvector.
          \end{enumerate} 
        \end{proof}

    \item \label{c}
        Laat $V$ een vectorruimte zijn van dimensie $n$ en $B$ een basis van $V$. (hier moet nog tekst). \newline
        We gaan bewijzen dat $f_B$ een isomorfisme geeft tussen $V$ en $\mathbb{R}_n$, in andere woorden dat $f_B$ een lineare en bijectieve functie is.

        We bewijzen eerst dat de functie linear is.

        \begin{proof}\leavevmode
            \begin{enumerate}[1.]
                \item 
                Bekijk $f_B(\mathbf{x} + \mathbf{y})$ met $\mathbf{x}, \mathbf{y} \in V$. Met de definitie van $f_B$ uit het dictaat zien we dat dit het volgende geeft:
                \begin{equation}
                    f_B(\mathbf{x} + \mathbf{y}) = (x_1 + y_1) \mathbf{b_1} + (x_2 + y_2) \mathbf{b_2} + \ldots + (x_n + y_n) \mathbf{b_n}.   
                \end{equation}
                Door distributiviteit van het scalair veelvoud over optelling in het lichaam kunnen we dit ook schrijven als:
                \begin{align*}
                    f_B(\mathbf{x} + \mathbf{y}) & = 
                    x_1 \mathbf{b_1} + y_1 \mathbf{b_1} + 
                    x_2 \mathbf{b_2} + y_2 \mathbf{b_2} + 
                    \ldots + x_n \mathbf{b_n} + y_n \mathbf{b_n} \\
                    & = 
                    x_1 \mathbf{b_1} + x_2 \mathbf{b_2} + \ldots + x_n \mathbf{b_n} +
                    y_1 \mathbf{b_1} + y_2 \mathbf{b_2} + \ldots + y_n \mathbf{b_n} \\
                    & = f_B(\mathbf{x}) + f_B(\mathbf{y}).
                \end{align*}

                \item 
                Bekijk nu $f_B(\lambda \mathbf{x})$ met $\mathbf{x}, \mathbf{y} \in V$. Vanuit de definitie van $f_B$ is dit te schrijven als:
                \begin{equation}
                    f_B(\lambda \mathbf{x})= 
                    \lambda x_1 \mathbf{b_1} + \lambda x_2 \mathbf{b_2} + \ldots + \lambda x_n \mathbf{b_n}
                \end{equation}
                Door distributiviteit van het scalair veelvoud over vector optelling kunnen we dit ook schrijven als:
                \begin{equation}
                    f_B(\lambda \mathbf{x})= 
                    \lambda ( x_1 \mathbf{b_1} + x_2 \mathbf{b_2} + \ldots + x_n \mathbf{b_n}) 
                    = \lambda f_B(\textbf{x}). 
                \end{equation}
            \end{enumerate}
            Uit deze twee punten concluderen we dat $f_B$ een lineare functie is.

        \end{proof}
        Nu bewijzen we dat $f_B$ een bijectie is.
        \begin{proof}\leavevmode
            \begin{enumerate}[1.]
                \item 
                We beginnen met bewijzen dat $f_B$ injectief is. Neem aan: 
                \begin{equation} \label{verg1}
                    f_B(\mathbf{x}) = f_B(\mathbf{y})
                \end{equation}
                
                met $\mathbf{x}, \mathbf{y} \in V$. Vanuit de definitie van $f_B$ is dit te schrijven als:
                \begin{equation}
                    x_1 \mathbf{b_1} + x_2 \mathbf{b_2} + \ldots + x_n \mathbf{b_n} = 
                    y_1 \mathbf{b_1} + y_2 \mathbf{b_2} + \ldots + y_n \mathbf{b_n}
                \end{equation}
                Door de rechterkant naar de linkerkant te halen, en door distributiviteit van het scalair veelvoud over optelling in het lichaam kunnen we dit ook schrijven als:
                \begin{equation}
                    (x_1 - y_1) \mathbf{b_1} + (x_2 - y_2) \mathbf{b_2} + 
                    \ldots + (x_n - y_n) \mathbf{b_n} = 0
                \end{equation}
                Per definitie zijn $\{\mathbf{b_1},\mathbf{b_2}, \ldots, \mathbf{b_n}\}$ onafhankelijk, dus volgens Stelling 7.2.1 uit het dictaat is de enige oplossing van deze vergelijking de triviale oplossing. Dus er geldt:
                \begin{align*}
                    x_1 & = y_1 \\
                    x_2 & = y_1 \\
                    & \ \: \vdots  \\
                    x_n & = y_n.
                \end{align*}
                Dus $f_B(\mathbf{x}) = f_B(\mathbf{y})$ impliceert $\mathbf{x} = \mathbf{y}$, dus $f_B$ is injectief.

                \item 
                We bewijzen nu dat $f_B$ surjectief is. Dit betekent dat voor elke $\mathbf{x}_B \in W$ er een $\mathbf{x} \in V$ bestaat zodat $f(\mathbf{x}) = \mathbf{x}_B$. We willen dus een $\mathbf{x}$ vinden zodat $f(\mathbf{x}) = \mathbf{x}_B$. We bewijzen dat de volgende $\mathbf{x}$ hieraan voldoet: 
                \begin{equation} 
                    \mathbf{x} = 
                    x_{B, 1} \mathbf{b}_1 + 
                    x_{B, 2} \mathbf{b}_2 + \ldots + 
                    x_{B, n} \mathbf{b}_n
                \end{equation}
                waar $x_{B, k}$ staat voor het $k$de element van de coördinatenkolom $\mathbf{x}_B$. Bekijk $f_B(\mathbf{x})$. We zien uit de definitie van $f_B$ dat:
                \begin{equation}
                    f_B(\mathbf{x}) = (x_{B, 1}, x_{B, 2}, \ldots , x_{B, n})^t
                \end{equation}
                De rechterzijde van de vergelijking is precies de vector $\mathbf{x}_B$, dus we zien dat:
                \begin{equation}
                    f_B(\mathbf{x}) = \mathbf{x}_B
                \end{equation}
                We kunnen dus voor elke $\mathbf{x}_B \in W$ een $\mathbf{x} \in V$ vinden zodat $f_B(\mathbf{x}) = \mathbf{x}_B$. Dus $f_B$ is surjectief.

            \end{enumerate}
            Deze twee punten maakt samen dat $f_B$ bijectief is.
        \end{proof}
        Dit samen maakt dat $f_B$ een lineare bijectieve functie is en dus een isomorfisme geeft tussen $V$ en $\mathbb{R}^n$. 

        \item \label{d}
        Neem $V = \mathbb{R}[X]_3$ en gegeven is $B = \{1 + X, X + X^2, X^2 + X^3, X^3 \}$ een basis van $V$. Neem ook $P_1(X) = 2 + 6X +3X^2 + 4X^3$. We laten zien dat $f_B(P_1(X)) = (2, 4, -1, 5)^t$.
        \begin{proof}
            Gegeven is: 
            \begin{equation}
                P_1(X) = 2 + 6X +3X^2 + 4X^3.
            \end{equation}
            We kunnen dit ook schrijven als:
            \begin{equation}
                P_1(X) = 2(1 + X) + 4(X + X^2) -1(X^2 + X^3) + 5(X^3).
            \end{equation}
            We hebben nu $P_1(X)$ geschreven als lineare combinatie van de gegeven basis $B$, dus we zien dat:
            \begin{equation}
                f_B(P_1(X)) = (2, 4, -1, 5)^t
            \end{equation}

        \end{proof}


\end{enumerate}
\end{document}