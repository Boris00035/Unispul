\documentclass[12pt, dutch, a4paper]{article}

\usepackage[hidelinks]{hyperref}
\usepackage[tmargin=0.8in, bmargin=1in]{geometry}
\usepackage{parskip}
\usepackage{amssymb}
\usepackage{amsmath}
\usepackage[shortlabels]{enumitem}
\usepackage[dutch]{babel}
\selectlanguage{dutch}
\usepackage{amsthm}
\usepackage{cleveref}
\usepackage{graphicx}
\usepackage{siunitx}

\theoremstyle{definition}

\newtheorem{theorem}{Stelling}
\newtheorem{lemma}{Lemma}[theorem]
\newtheorem{lemmalos}{Lemma}
\newtheorem{sublemma}{Lemma}[lemma]
\newtheorem{case}{Geval}
\newtheorem{claim}{Claim}

\newenvironment{shortclaim}
  {\refstepcounter{claim}\textbf{Claim~\theclaim.}}% \begin{shortthm}
{\enskip}
\newenvironment{shortthm}
  {\refstepcounter{theorem}\textbf{Stelling~\thetheorem.}}% \begin{shortthm}
{\enskip}

\usepackage[newfloat]{minted}
\usepackage[
  skip=0pt
  ]{caption}

\newenvironment{code}{\captionsetup{type=listing}}{}
\SetupFloatingEnvironment{listing}{placement=htp}
\SetupFloatingEnvironment{listing}{name=Code blok}

% \usepackage{mathtools}
% \newcommand\SetSymbol[1][]{\nonscript\:#1\vert\allowbreak\nonscript\:\mathopen{}}
% \providecommand\given{} % to make it exist
% \DeclarePairedDelimiterX\Set[1]\{\}{\renewcommand\given{\SetSymbol[\delimsize]}#1}


\title{Inlever 2 Lial 2}
\author{Boris van Boxtel en Lotte Gritter}
\date{21 December 2022}

\begin{document}

\maketitle 

\begin{enumerate}[\text{Opgave} 1.]
    \item \label{opdracht1}
    Gegeven zijn twee eindig dimensionale vectorruimten 
    $V$ en $W$ en een lineare afbeelding $A\colon V \to W$. 
    Er wordt bewezen dat als $\dim(V) > \dim(W)$, $A$ niet injectief is.
    
    \begin{proof} $ $ \newline
        Volgens de dimensiestelling geldt de volgende vergelijking:
        \begin{equation}
            \dim(V) = \dim(\ker(A)) + \dim(A(V))
        \end{equation}
        Hieruit volgt de volgende vergelijking:
        \begin{equation} \label{dimensiestelling}
            \dim(V) - \dim(A(V)) = \dim(\ker(A)).
        \end{equation}
        Per definitie geldt het volgende: 
        (komt omdat $A(V) \subseteq W$, moet meer uitleg bij?)
        \begin{equation}
            \dim(W) \geq \dim(A(V))
        \end{equation}
        En samen met het gegeven dat $\dim(V) > \dim(W)$,
        gelden de volgende ongelijkheden:
        \begin{equation}
            \dim(V) > \dim(W) \geq \dim(A(V))
        \end{equation}
        \begin{equation}
            \dim(V) > \dim(A(V))
        \end{equation}
        Hieruit volgt dat
        \begin{equation}
            \dim(V) - \dim(A(V)) > 0.
        \end{equation}
        Vanuit \cref{dimensiestelling} volgt dus dat $\dim(\ker(A)) > 0$.
        Samen met \textbf{Stelling 7.3.3} 
        is de conclusie dat $A$ niet injectief is. 

    \end{proof}
    
    \item \label{opdracht2}
    Het doel is de oplossingsverzameling van de vergelijking
    \begin{equation}\label{maineq}
        2x^2 - 4xy + 5y^2 = 36
    \end{equation}
    om te schrijven naar een vorm 
    \begin{equation}
        a(x')^2 + b(y')^2 = 36
    \end{equation}
    met behulp van lineare algebra. We geven eerst een beeld van de aanpak om dit bereiken. 
    
    We beginnen met het schrijven van \cref{maineq} in de volgende vorm:
    \begin{equation}\label{quadform}
        \mathbf{x}^T A \mathbf{x} = 36
    \end{equation}
    Waar $\mathbf{x} = (x, y)^T$ en $A$ een $2\times 2$ matrix, namelijk 
    $A = \begin{pmatrix}
        2 & -2 \\
        -2 & 5
    \end{pmatrix}$. We zoeken nu een basis van eigenvectoren van $A$, zodat we $A$ kunnen diagonaliseren. Wanneer we deze hebben gevonden, kan $A$ geschreven worden als $A = S^{-1} D S$. Voor een coördinatentransformatie matrix $S$ en een diagonaalmatrix $D$. Als we dit invullen in \cref{quadform} vinden we de volgende vergelijking: 
    \begin{equation}\label{diagonalized}
        \mathbf{x}^T S^{-1} D S \mathbf{x} = 36.
    \end{equation} 
    Vervolgens bewijzen we dat $S^{-1} = S^T$, waardoor we \cref{diagonalized} kunnen schrijven als:
    \begin{equation}
        \mathbf{x}^T S^T D S \mathbf{x} = 36.
    \end{equation} 
    wat we volgens \textbf{Stelling 3.1.1} kunnen schrijven als volgt:
    \begin{equation}
        (S \mathbf{x})^T D S \mathbf{x} = 36.
    \end{equation}
    Hier zien we in dat we aan de linker en rechterkant nu nieuwe coördinaten $S\mathbf{x} = \mathbf{x'} = (x', y')^T$ hebben, dus we vinden het volgende:
    \begin{equation} \label{transed_eq}
        \mathbf{x'}^T D \mathbf{x'} = 36.
    \end{equation}
    Per definitie is $D$ een diagonaal matrix, dus vinden we een polynoom in de coördinaten $(x', y')^T$ zonder kruistermen. Nu volgt de uitvoering voor de gegeven vergelijking.

    Voor een basis van eigenvectoren van $A$ beginnen we bij het zoeken van eigenwaarden van $A$. Dit doen we door de vergelijking $\det(A - \lambda I) = 0$ op te lossen. Dit geeft de volgende vergelijking in $\lambda$:
    \begin{equation}
        (2 - \lambda)(5 - \lambda) -4 = 0
    \end{equation}
    De oplossingen hiervan zijn $\lambda_1 = 1, \lambda_2 = 6$, met beide $\lambda_1, \lambda_2$ een algebraïsche multipliciteit van 1. We vinden de eigenvectoren bij de eigenwaarden $\lambda_1$ en  $\lambda_2$ door een basis te vinden van $\ker (A - \lambda_1 I) = E_1$ en $\ker (A - \lambda_2 I) = E_2$ respectievelijk. Dit doen we door Gauss-Jordan toe te passen op de matrix $A - \lambda I$, en vervolgens een basis op te stellen. We vinden de volgende bases van de eigenruimtes:
    \begin{align}
        B_1 &= \left\{ 
            \begin{pmatrix}
            \frac{2}{\sqrt{5}} \\
            \frac{1}{\sqrt{5}}
            \end{pmatrix}\right\}
        \\
        B_6 &= \left\{ 
            \begin{pmatrix}
            -\frac{1}{\sqrt{5}} \\
            \frac{2}{\sqrt{5}}
            \end{pmatrix}\right\}
    \end{align}
    Waar $B_1$ de basis is van $E_1$ en $B_6$ de basis van $E_6$. Deze specifieke vectoren zijn zo gekozen zodat de lengte 1 is. We zien dat deze bases beide uit 1 vector bestaan. Dus geldt dat de dimensie van $E_1$ en $E_6$ beide gelijk is aan 1. Dus voor elke eigenwaarde geldt dat de algebraïsche multipliciteit gelijk is aan de meetkundige multipliciteit, dus $A$ is diagonalizeerbaar. We zien dat het inproduct van de twee vectoren in $B_1$ en $B_6$ 0 is, dus ze staan loodrecht op elkaar en vormen dus een basis van $\mathbb{R}^n$. De coördinatentransformatie matrix $S$ en de diagonaal matrix $D$ zijn gegeven door:
    \begin{align}
        S &= 
        \begin{pmatrix}
            \frac{2}{\sqrt{5}} & \frac{1}{\sqrt{5}} \\
            -\frac{1}{\sqrt{5}} & \frac{2}{\sqrt{5}}
        \end{pmatrix} &
        D &= 
        \begin{pmatrix}
            1 & 0 \\
            0 & 6
        \end{pmatrix} 
    \end{align}   
    We laten nu zien dat $S^{-1} = S^T$.
    \begin{proof} $ $ \newline
        We kunnen beide matrices berekenen, en vinden het volgende.
        \begin{align}
            S^{-1} &= 
            \begin{pmatrix}
                \frac{2}{\sqrt{5}} & -\frac{1}{\sqrt{5}} \\
                \frac{1}{\sqrt{5}} & \frac{2}{\sqrt{5}}
            \end{pmatrix} & 
            S^T &= 
            \begin{pmatrix}
                \frac{2}{\sqrt{5}} & -\frac{1}{\sqrt{5}} \\
                \frac{1}{\sqrt{5}} & \frac{2}{\sqrt{5}}
            \end{pmatrix} 
        \end{align}
        We zien dus dat voor deze $S$ geldt dat $S^{-1} = S^T$.

    \end{proof}
    
    We hebben alle benodigde componenten gevonden. Als we de gevonden matrix voor $D$ invullen in \cref{transed_eq} vinden we het volgende:
    \begin{equation}
        \mathbf{x'}^T 
        \begin{pmatrix}
            1 & 0 \\
            0 & 6
        \end{pmatrix} 
        \mathbf{x'} = 36
    \end{equation}
    en dit kunnen we schrijven als:
    \begin{equation} \label{transed_eq2}
        {x'}^2 + 6{y'}^2 = 36.
    \end{equation}
    We hebben de orginele vraag beantwoord met oplossing $a = 1$ en $b = 6$.

    Ook zien we dat de volgende vergelijking geldt: 
    \begin{equation}
        S\mathbf{x} = 
        \mathbf{x'} = 
        \begin{pmatrix}
            x' \\
            y'
        \end{pmatrix}
        = \frac{1}{\sqrt{5}}
        \begin{pmatrix}
            2x + y \\
            -x + 2y
        \end{pmatrix}.
    \end{equation}
    Als we ter controle deze twee vergelijking van $x'$ en $y'$ invullen in \cref{transed_eq2}, vinden we zoals verwacht \cref{maineq} terug.
\end{enumerate}
\end{document}

