\documentclass[12pt, a4paper]{article}

\usepackage[hidelinks]{hyperref}
\usepackage[tmargin=0.8in, bmargin=0.8in]{geometry}
\usepackage{parskip}
\usepackage{amssymb}
\usepackage{amsmath}
\usepackage[shortlabels]{enumitem}
\usepackage[dutch]{babel}
\selectlanguage{dutch}
\usepackage{pgfplots}
\pgfplotsset{width=7.5cm, compat=1.18}
\usepackage{amsthm}

\begin{document}

\title{Mechanica - Inleveropgave 1}
\author{Boris van Boxtel - Jean Croes - Mike - Lotte Gritter}
\date{28 september 2022 - Week 39}

\maketitle
\pagenumbering{gobble}

\begin{enumerate}[(a).]
    \item \label{itema}
    $(x,y, t) = (L ,h ,\frac{h}{c})$ of $(x,y, t) = (L ,-h ,\frac{h}{c})$.

    \item
    Het toepassen van een Lorentz transformatie op de bij \ref{itema} 
    gevonden ruimte-tijd coördinaten, 
    geeft als coördinaten in stelsel $\mathcal{O}^{'}$:
    $(L,\: \gamma h(1-\frac{v}{c}),\: \gamma \frac{h}{c}(1-\frac{v}{c}))$ en 
    $(L, \:-\gamma h(1+\frac{v}{c}), \: \gamma \frac{h}{c}(1+\frac{v}{c}))$.

    \item 
    De tijd die het licht erover doet om van het lampje naar het papiertje 
    te gaan is $\frac{h}{c}$. De tijd die het licht erover doet om van het papiertje 
    naar de oorsprong te gaan is $\frac{\sqrt{c^2 + h^2}}{c}$. 
    Het tijdstip waarop de waarnemer in $\mathcal{O}$ de lichtflits via 
    het papiertje ziet wordt dus gegeven door:
    \begin{equation} \label{tijdinO} 
        t = \frac{h}{c} + \frac{\sqrt{c^2 + h^2}}{c} 
    \end{equation}

    \item \label{itemd}
    Door lengtecontractie is $h' \neq h$, dus is de tijd die het licht erover doet 
    om bij de oorsprong te komen anders. Door een lorentztransformatie toe te passen
    vinden we:
    \begin{equation} \label{lorentzh}
        h' = \gamma h\left( 1-\beta\right) 
    \end{equation}
    Waar $\beta = \frac{v}{c}$.

    Nu kunnen we in \eqref{tijdinO} de substitutie $h = h'$ maken om de tijd 
    gemeten door $\mathcal{O}^{'}$ te vinden. Hieruit volgt dat het tijdstip 
    waarop $\mathcal{O}$ de lichtflits waarneemt gegeven wordt door:
    \begin{equation} \label{tijdinO'} 
        t' = \frac{\gamma h\left( 1-\frac{v}{c}\right)}{c} + 
        \frac{\sqrt{c^2 + (\gamma h\left( 1-\frac{v}{c}\right))^2}}{c}
    \end{equation}

    \item \label{iteme}
    De afstand van het lampje tot het papiertje volgens de waarnemer in $\mathcal{O}'$ 
    is $h'$. En de afstand van het papiertje tot de oorsprong van $\mathcal{O}'$ is 
    $\sqrt{L^2 + (vt-h')^2}$. Dus de afgelegde afstand van het licht is 
    $\sqrt{L^2 + (vt-h')^2} + h'$. 
    De tijd die het licht hierover doet is dus gegeven door:
    \begin{equation}
        \Delta t' = \frac{\sqrt{L^2 + (vt-h')^2} + h'}{c}
    \end{equation}
    En met (\ref{lorentzh}) zien we dan dit gelijk is aan de volgende vergelijking:
    \begin{equation}
        \Delta t' = \frac{\sqrt{L^2 + (vt-\gamma h\left( 1-\beta\right))^2} +
        \gamma h\left( 1-\beta\right)}{c}
    \end{equation}
    Voor een bepaalde $t$.

    \item 
    De antwoorden van \ref{itemd} en \ref{iteme} zijn niet hetzelfde, 
    omdat de afstand die het licht aflegt naar $\mathcal{O}'$ niet voor elke $t$
    hetzelfde is als de afgelegde afstand naar $\mathcal{O}$.
\end{enumerate}


\end{document}