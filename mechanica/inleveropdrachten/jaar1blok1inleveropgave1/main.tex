\documentclass[12pt, a4paper]{article}

\usepackage[hidelinks]{hyperref}
\usepackage[tmargin=0.8in, bmargin=0.8in]{geometry}
\usepackage{parskip}
\usepackage{amssymb}
\usepackage{amsmath}
\usepackage[shortlabels]{enumitem}
\usepackage[dutch]{babel}
\selectlanguage{dutch}
\usepackage{pgfplots}
\pgfplotsset{width=7.5cm, compat=1.18}
\usepackage{amsthm}

\begin{document}

\title{Mechanica - Inleveropgave 1}
\author{Boris van Boxtel - Jean Croes - Mike van Eck - Lotte Gritter}
\date{28 september 2022 - Week 39}

\maketitle
\pagenumbering{gobble}

\begin{enumerate}[(a).]
    \item \label{itema}
    De ruimtetijdcoördinaten van de events in stelsel O zijn:
    $(x,y, ct) = (L ,h ,h)$ en $(x,y, ct) = (L ,-h ,h)$.

    \item
    Het toepassen van een Lorentztransformatie op de bij \ref{itema} 
    gevonden ruimte-tijd coördinaten, 
    geeft als coördinaten in stelsel $\mathcal{O}^{'}$:
    $(L,\: \gamma h(1-\frac{v}{c}),\: \gamma h(1-\frac{v}{c}))$ en 
    $(L, \:-\gamma h(1+\frac{v}{c}), \: \gamma h(1+\frac{v}{c}))$. 
    Waar $v$ de snelheid van $\mathcal{O}'$ is ten opzichte van 
    $\mathcal{O}$.

    \item 
    De tijd die het licht erover doet om van het lampje naar het papiertje 
    te gaan is $\frac{h}{c}$. De tijd die het licht erover doet om van het papiertje 
    naar de oorsprong te gaan is $\frac{\sqrt{L^2 + h^2}}{c}$. 
    Het tijdstip waarop de waarnemer in $\mathcal{O}$ de lichtflits via 
    het papiertje ziet wordt dus gegeven door:
    \begin{equation} \label{tijdinO} 
        t = \frac{h}{c} + \frac{\sqrt{L^2 + h^2}}{c} 
    \end{equation}

    \item \label{itemd}
    Door lengtecontractie is $h' \neq h$, dus is de tijd die het licht erover doet 
    om bij de oorsprong te komen anders. Door een lorentztransformatie toe te passen
    vinden we:
    \begin{equation} \label{lorentzh}
        h' = \gamma h\left( 1-\beta\right) 
    \end{equation}
    Waar $\beta = \frac{v}{c}$.

    Nu kunnen we in \eqref{tijdinO} de substitutie $h = h'$ maken om de tijd 
    gemeten door $\mathcal{O}^{'}$ te vinden. Hieruit volgt dat het tijdstip 
    waarop $\mathcal{O}$ de lichtflits waarneemt gegeven wordt door:
    \begin{equation} \label{tijdinO'} 
        t' = \frac{\gamma h\left( 1-\frac{v}{c}\right)}{c} + 
        \frac{\sqrt{L^2 + \left(\gamma h\left( 1-\frac{v}{c}\right)\right)^2}}{c}
    \end{equation}

    \item \label{iteme}
    % De afstand van het lampje tot het papiertje volgens de waarnemer in $\mathcal{O}'$ 
    % is $h'$. En de afstand van het papiertje tot de oorsprong van $\mathcal{O}'$ is 
    % $\sqrt{L^2 + (vt \pm h')^2}$. Dus de afgelegde afstand van het licht is 
    % $\sqrt{L^2 + (vt \pm h')^2} + h'$. 
    % De tijd die het licht hierover doet is dus gegeven door:
    % \begin{equation}
    %     \Delta t' = \frac{\sqrt{L^2 + (vt \pm h')^2} + h'}{c}
    % \end{equation}
    % En met (\ref{lorentzh}) zien we dan dit gelijk is aan de volgende vergelijking:
    % \begin{equation}
    %     \Delta t' = \frac{\sqrt{L^2 + (vt \pm \gamma h\left( 1-\beta\right))^2} +
    %     \gamma h\left( 1-\beta\right)}{c}
    % \end{equation}
    % Voor een bepaalde $t$.
    Volgens de waarnemer in $\mathcal{O}$ is de tijd die het licht erover doet om bij
    het papiertje aan te komen gelijk aan $\frac{h}{c}$. 
    De snelheid van $\mathcal{O}'$ ten opzichte van $\mathcal{O}$ is $v$,
    dus $v \Delta t$ is de afgelegde afstand over een tijd $\Delta t$.
    Dus de tijd die het licht erover doet om van het papiertje naar de oorsprong van
    $\mathcal{O}$ te komen, is gelijk aan $\frac{\sqrt{L^2 + (vt \pm h)}}{c}$.
    Dus de tijd die het licht erover doet om in $\mathcal{O}'$ aan te komen 
    wordt gegeven door:
    \begin{equation}
        t = \frac{h}{c} + \frac{\sqrt{L^2 + \left(v \Delta t \pm h\right)^2}}{c}
    \end{equation}
    
    \newpage
    Omdat de tijd die het licht erover doet om in $\mathcal{O}'$ aan te komen,
    en de tijd die $\mathcal{O}'$ erover doet om in dat punt te komen gelijk zijn,
    geldt $t = \Delta t$. Dus de volgende vergelijking geldt:

    \begin{equation}
        t = \frac{h}{c} + \frac{\sqrt{L^2 + \left(vt \pm h\right)^2}}{c}
    \end{equation}

    Deze vergelijking resulteert in een kwadratische vergelijking in $t$, 
    die kan worden opgelost om de volgende oplossingen te vinden:

    \begin{equation}
        \begin{split}
            t_1 = \frac{h}{v+c}\ +\sqrt{\frac{h^{2}}{\left(v+c\right)^{2}}-\frac{L^{2}}{v^{2}-c^{2}}}\\
            t_2 = -\frac{h}{v-c}+\sqrt{\frac{h^{2}}{\left(v-c\right)^{2}}-\frac{L^{2}}{v^{2}-c^{2}}}  
        \end{split}
    \end{equation}
    
    Waar we de oplossingen die een negatieve waarden voor $t$ aannemen buiten beschouwing
    hebben gelaten.

    Dit is de tijd wanneer dat het licht in $\mathcal{O}'$ aankomt volgens een waarnemer in $\mathcal{O}$.
    Dit moeten we dus nog vertalen naar een tijd in $\mathcal{O}'$.
    Omdat het event waar het licht en de oorsprong van $\mathcal{O}'$ samenkomen per definitie
    in de oorsprong van $\mathcal{O}'$ ligt, 
    is de tijd waarop dit gebeurt een eigentijd van $\mathcal{O}'$.
    Dus geldt $\tau' = t / \gamma$ waar $\tau'$ de eigentijd van $\mathcal{O}'$ is.
    
    Dus het tijdstip waarop de waarnemer in $\mathcal{O}'$
    de lichtflits via de papiertjes waarneemt, is gegeven door:
    \begin{equation}
        \begin{split}
            t'_1 = \sqrt{1 - \frac{v^2}{c^2}} \left[\frac{h}{v+c} +
            \sqrt{\frac{h^{2}}{\left(v+c\right)^{2}}-\frac{L^{2}}{v^{2}-c^{2}}}\right]\\
            t'_2 = \sqrt{1 - \frac{v^2}{c^2}} \left[-\frac{h}{v-c}+
            \sqrt{\frac{h^{2}}{\left(v-c\right)^{2}}-\frac{L^{2}}{v^{2}-c^{2}}}\right]
        \end{split}
    \end{equation}
    En dit kunnen we herschrijven tot het volgende:
    \begin{equation}
        \begin{split}
            t'_1 = \frac{h}{c}\sqrt{\frac{c-v}{c+v}}+
            \frac{1}{c}\sqrt{L^{2} + h^{2}\left(\frac{c-v}{c+v}\right)}\\
            t'_2 = \frac{h}{c}\sqrt{\frac{\left(c+v\right)}{c-v}}+
            \frac{1}{c}\sqrt{L^{2} + h^{2}\frac{\left(c+v\right)}{\left(c-v\right)}}
        \end{split}
    \end{equation}

    \item 
    
    Bij \ref{itemd} en \ref{iteme} worden verschillende dingen beschreven. 
    Bij \ref{itemd} beschrijf je de tijd die het licht er, volgens de waarnemer in 
    $\mathcal{O}'$, in $\mathcal{O}$ over doet om de waarnemer in de oorsprong van 
    $\mathcal{O}$ te bereiken. 
    Dit is korter dan de tijd die de waarnemer in $\mathcal{O}$ waarneemt omdat 
    de afstand $h$ tot de papiertjes volgens de waarnemer in $\mathcal{O}'$ 
    korter is vanwege lengtecontractie. 
    
    Bij \ref{iteme} wordt de tijd beschreven die het licht erover doet om 
    via de papiertjes bij de waarnemer in de oorsprong van $\mathcal{O}'$ te komen. 
    Dit is iets heel anders en geeft dus ook een andere tijd.
    
    
\end{enumerate}


\end{document}