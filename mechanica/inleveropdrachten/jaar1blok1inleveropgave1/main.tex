\documentclass[12pt, dutch, a4paper]{article}

\usepackage[hidelinks]{hyperref}
\usepackage[tmargin=0.8in, bmargin=1in]{geometry}
\usepackage{parskip}
\usepackage{amssymb}
\usepackage{amsmath}
\usepackage[shortlabels]{enumitem}
\usepackage[dutch]{babel}
\selectlanguage{dutch}
\usepackage{pgfplots}
\pgfplotsset{width=7.5cm, compat=1.18}
\usepackage{amsthm}
\usepackage{cleveref}

\begin{document}

\title{Mechanica - Inleveropgave 1}
\author{Boris van Boxtel - Jean Croes - Mike van Eck - Lotte Gritter}
\date{28 september 2022 - Week 39}

\maketitle
\pagenumbering{arabic}

Gegeven zijn twee stelsels: $\mathcal{O}$ en $\mathcal{O'}$, waarbij $\mathcal{O'}$ met een snelheid $v$
in y-richting beweegt. Afgesproken is dat op $t=0$ de oorsprong van $\mathcal{O'}$ in de oorsprong 
van $\mathcal{O}$ ligt.
\bigskip

\begin{enumerate}[(a).]
    \item \label{itema} 
    De tijd kan berekend worden door de afstand te delen door de tijd. 
    De tijd is dus gelijk aan $\frac{h}{c}$. Hieruit volgt dat $ct=h$.
    De ruimtetijdcoördinaten van de events in stelsel $\mathcal{O}$ zijn:
    $(x,y, ct) = (L ,h ,h)$ en $(x,y, ct) = (L ,-h ,h)$.

    \item
    Het toepassen van een lorentztransformatie op de bij \ref{itema} 
    gevonden ruimte-tijd coördinaten, 
    geeft als coördinaten in stelsel $\mathcal{O}^{'}$:
    $(L,\: \gamma h(1-\frac{v}{c}),\: \gamma h(1-\frac{v}{c}))$ en 
    $(L, \:-\gamma h(1+\frac{v}{c}), \: \gamma h(1+\frac{v}{c}))$. 
    Omdat $\mathcal{O}'$ uitsluitend in de y-richting beweegt,
    verandert alleen het y-coördinaat na het toepassen van de lorentztransformatie.  

    \item \label{itemc}
    De tijd die het licht erover doet om van het lampje naar het papiertje 
    te gaan is $\frac{h}{c}$. 
    De afstand van het papiertje tot de oorsprong van $\mathcal{O}$ is te
    berekenen met de stelling van Pythagoras. Dit geeft $\sqrt{L^2 + h^2}$.
    De tijd die het licht erover doet om van het papiertje 
    naar de oorsprong te gaan is dan $\frac{\sqrt{L^2 + h^2}}{c}$. 
    Het tijdstip waarop de waarnemer in $\mathcal{O}$ de lichtflits via 
    het papiertje ziet wordt dus gegeven door: 
    \begin{equation} \label{tijdinO} 
        t_{\mathcal{O}} = \frac{h}{c} + \frac{\sqrt{L^2 + h^2}}{c} 
    \end{equation}

    Waar het subscript staat voor de oorsprong van het stelsel 
    waar de lichtflits heen beweegt.

    \item \label{itemd}
    Doordat stelsel $\mathcal{O'}$ in de y-richting beweegt, is $h' \neq h$, dus is de tijd die 
    het licht erover doet om in de oorsprong van $\mathcal{O'}$ te komen anders. Door een 
    lorentztransformatie toe te passen vinden we:
    \begin{equation} \label{lorentzh}
        h' = \gamma h\left( 1-\beta\right) 
    \end{equation}
    Waar $\beta = \frac{v}{c}$.

    Nu kunnen we in \eqref{tijdinO} $h$ vervangen door $h'$ om de tijd 
    gemeten door $\mathcal{O}^{'}$ te vinden. Hieruit volgt dat het tijdstip 
    waarop de waarnemer in $\mathcal{O}$ de lichtflits waarneemt volgens $\mathcal{O'}$ gegeven wordt door:
    \begin{equation} \label{tijdinO'} 
        t'_\mathcal{O} = \frac{\gamma h\left( 1-\frac{v}{c}\right)}{c} + 
        \frac{\sqrt{L^2 + \left(\gamma h\left( 1-\frac{v}{c}\right)\right)^2}}{c}
    \end{equation}

    \item \label{iteme}
    Volgens de waarnemer in $\mathcal{O}$ is de tijd die het licht erover doet om bij
    het papiertje aan te komen gelijk aan $\frac{h}{c}$. 
    De snelheid van $\mathcal{O}'$ ten opzichte van $\mathcal{O}$ is $v$,
    dus $v \Delta t$ is de afgelegde afstand over een tijd $\Delta t$.
    Dus de tijd die het licht erover doet om van het papiertje naar de oorsprong van
    $\mathcal{O}$ te komen, is gelijk aan $\frac{\sqrt{L^2 + (vt \pm h)}}{c}$.
    Dus de tijd die het licht erover doet om in $\mathcal{O}'$ aan te komen 
    wordt gegeven door:
    \begin{equation}
        t_{\mathcal{O}'}= \frac{h}{c} + \frac{\sqrt{L^2 + \left(v \Delta t \pm h\right)^2}}{c}
    \end{equation}
    
    Omdat de tijd die het licht erover doet om in $\mathcal{O}'$ aan te komen,
    en de tijd die $\mathcal{O}'$ erover doet om in dat punt te komen gelijk zijn,
    geldt $t = \Delta t$. Dus de volgende vergelijking geldt:
    \begin{equation}
        t_{\mathcal{O}'} = \frac{h}{c} + \frac{\sqrt{L^2 + \left(vt \pm h\right)^2}}{c}
    \end{equation}

    Deze vergelijking resulteert in een kwadratische vergelijking in $t$, 
    die kan worden opgelost om de volgende oplossingen te vinden:
    \begin{equation}
        \begin{split}
            t_{\mathcal{O}',1} = \frac{h}{c + v}+\sqrt{\frac{h^{2}}{\left(c + v\right)^{2}}+\frac{L^{2}}{c^{2}-v^{2}}}\\
            t_{\mathcal{O}',2} = -\frac{h}{c - v}+\sqrt{\frac{h^{2}}{\left(c - v\right)^{2}}+\frac{L^{2}}{c^{2}-v^{2}}}  
        \end{split}
    \end{equation}
    
    Waar we de oplossingen die een negatieve waarden voor $t$ aannemen buiten beschouwing
    hebben gelaten.

    Dit is de tijd wanneer dat het licht in $\mathcal{O}'$ aankomt volgens een waarnemer in $\mathcal{O}$.
    Dit moeten we dus nog vertalen naar een tijd in $\mathcal{O}'$.
    Omdat het event waar het licht en de oorsprong van $\mathcal{O}'$ samenkomen per definitie
    in de oorsprong van $\mathcal{O}'$ ligt, 
    is de tijd waarop dit gebeurt volgens $\mathcal{O}'$ de eigentijd van dit event.
    Dus geldt $\tau' = t / \gamma$ 
    waar $\tau'$ de eigentijd van het event van de waarneming van het licht is in $\mathcal{O}'$, dus
    $\tau' = t'$.

    Dus het tijdstip waarop de waarnemer in $\mathcal{O}'$
    de lichtflits via de papiertjes waarneemt, is gegeven door:
    \begin{equation} 
        \begin{split}
            t'_{\mathcal{O}',1} = \sqrt{1 - \frac{v^2}{c^2}} \left[\frac{h}{c + v} +
            \sqrt{\frac{h^{2}}{\left(c + v\right)^{2}}+\frac{L^{2}}{c^{2}- v^{2}}}\: \right]\\
            t'_{\mathcal{O}',2} = \sqrt{1 - \frac{v^2}{c^2}} \left[\frac{h}{c-v}+
            \sqrt{\frac{h^{2}}{\left(c-v\right)^{2}}+\frac{L^{2}}{c^{2}-v^{2}}}\: \right]
        \end{split}
    \end{equation} 
    En dit kunnen we herschrijven tot het volgende:
    \begin{equation} \label{eq8}
        \begin{split}
            t'_{\mathcal{O}',1} = \frac{h}{c}\sqrt{\frac{c-v}{c+v}}+
            \frac{1}{c}\sqrt{L^{2} + h^{2}\left(\frac{c-v}{c+v}\right)}\\
            t'_{\mathcal{O}',2} = \frac{h}{c}\sqrt{\frac{c+v}{c-v}}+
            \frac{1}{c}\sqrt{L^{2} + h^{2}\left(\frac{c+v}{c-v}\right)}
        \end{split}
    \end{equation}
    En dit is ook weer te herschrijven als:
    \begin{equation}
        \begin{split}
            t'_{\mathcal{O}',1} &= \frac{h}{c} \, k(-\beta)+
            \frac{1}{c}\sqrt{L^{2} + h^2\,k(-\beta)^2}\\
            t'_{\mathcal{O}',2} &= \frac{h}{c} \, k(\beta)+
            \frac{1}{c}\sqrt{L^{2} + h^2\,k(\beta)^2}
        \end{split}
    \end{equation}
    Waar $k(\beta) = \sqrt{\frac{1+\beta}{1-\beta}}$.

    Om te testen of we de goede formule hebben, kunnen we $v = 0$ in \cref{eq8} invullen, en dan zouden 
    we dezelfde vergelijking moeten krijgen als \cref{tijdinO}.

    Als we $v = 0$ nemen geeft dit:
    \begin{equation}
        \begin{split}
            t'_{\mathcal{O}',1} = \frac{h}{c}\sqrt{\frac{c-0}{c+0}}+
            \frac{1}{c}\sqrt{L^{2} + h^{2}\left(\frac{c-0}{c+0}\right)}\\
            t'_{\mathcal{O}',2} = \frac{h}{c}\sqrt{\frac{c+0}{c-0}}+
            \frac{1}{c}\sqrt{L^{2} + h^{2}\left(\frac{c+0}{c-0}\right)}
        \end{split}
    \end{equation}
    Omdat $t'_1$ en $t'_2$ samenvallen, geven deze allebei:
    \begin{equation}
        t'_{\mathcal{O}'} = \frac{h}{c}+
        \frac{1}{c}\sqrt{L^{2} + h^{2}} = \frac{h}{c} + \frac{\sqrt{L^2 + h^2}}{c}
    \end{equation}

    Wat hetzelfde is als \cref{tijdinO}.

    \item 
    Bij \ref{itemd} en \ref{iteme} worden verschillende dingen beschreven. 
    Bij \ref{itemd} beschrijf je de tijd die het licht er, volgens de waarnemer in 
    $\mathcal{O}'$, in $\mathcal{O}$ over doet om de waarnemer in de oorsprong van 
    $\mathcal{O}$ te bereiken. 
    Dit is korter dan de tijd die de waarnemer in $\mathcal{O}$ waarneemt omdat 
    de afstand $h$ tot de papiertjes volgens de waarnemer in $\mathcal{O}'$ 
    korter is vanwege lengtecontractie. 
    
    Bij \ref{iteme} wordt de tijd beschreven die het licht erover doet om 
    via de papiertjes bij de waarnemer in de oorsprong van $\mathcal{O}'$ te komen. 
    Dit is iets heel anders en geeft dus ook een andere tijd.
    
    
\end{enumerate}
\end{document}